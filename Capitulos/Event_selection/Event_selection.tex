\chapter{Event Selection Criteria}
\label{Event_selection_criteria_chapter}

The analysis of the simulated data started by imposing a minimum value of 20.0 GeV on $p_T$ for b-jets and taus, and of 30.0 on jets that are not associated to taus or b quarks. The low values required on $p_T$ are imposed in order to preselectionate the interesting particles. After this was done, we required a minimal distance of separation between taus and muons, taus and electrons, and taus and jets. The former was because, we have to be sure that taus do not overlap with other objects so the performed analysis can evade to commit errors such as counting a particle two times. To do this, one has to impose a minimal value on the variable called minimal separation distance R, which is given by the Equation \ref{minimal_separation}. The minimal value imposed on this variable was 0.3 for taus with muons, electrons and jets. 

\begin{equation}
R \equiv \sqrt{\Delta \eta ^2 + \Delta \phi ^2}
\label{minimal_separation}
\end{equation}

Since we considered that the event of interest is generated by a procces of VBF, in the analysis was necessary to find the two possible jets that are related to the two VBF jets. This was done by searching two jets that satisfy the conditions of having a minimal value on $p_T$ of 50.0 GeV and $\eta$ greater that 5.0 each. The condition of minimal values on $p_T$ and $\eta$, is required because it is expected that the VBF jets have a large momentum and that they are located at the end-caps of the detector. Then, an algorithm was performed to select the pair of jets that have the largest sum of the mass of all possible combinations of jets pairs in each event. Finally, a minimal value of 100 MeV is required for the sum of the jets. That is how the two jets from the VBF process are determined. From both jets, the one with the hightest momentum is referred as the leading jet, and the other is know as the subleading jet. The required values to identificate the VBF jets are part of the preselection process.

After the former preselection process is made, we have to find the minimal or maximum values of variables that allows to reduce the background of the signal at maximum. These minimal and maximum required values are known as cuts. The first cut imposed in the analysis was on the number of jets. It was required a maximum number of 5 jets because the signal is expected to have 4 jets, an additional jet is included in order to find the best two VBF jets. The second cut imposed was the presence of just one tau, due to the fact that this is a characteristic expected for the hadronic signal. 

Next, the third cut is imposed by requiring the number of b-jets to be zero. The former is done to reduce drastically the $t\overline{t}$ background because it has in the final state b-jets. The following cut made was on the variable $\vec{E_T^{miss}}$, it was required a value greater that 20 MeV. The former is because in the hadronic signal the neutrino is the only source of $\vec{E_T^{miss}}$, thus it is expected that this variable have a low value.
  
After it was imposed a cut on the $\vec{E}_T^{miss}$ variable, the cuts motivated by the VBF proccess were performed. The first cut guarantees that there are minimum two jets that sasisfy the condition on $p_T$ to be candidates of VBF jets. Next, it is required that the multiplication of eta of both jets is negative, this implies that both jets are located at the opposite end-caps, as it is expected for both of the VBF jets. Next, it is imposed a minimal value of the diference in $\eta$ for both jets, referred as $\Delta \eta$. The minimal value on $\Delta eta$ was required to be 3.8, since the VBF jets should have a large difference in the pseudorapidity value. Finally, a cut in the sum of the mass of the VBF jets, known as dijet mass, was made. It was imposed a value of minimum 500.0 MeV on the dijet mass, due to the fact that the VBF jets have a large momentum, and this is proportional to the mass.  

The table \ref{preselection_table} shows the preselection values imposed that were mentioned on the initial paragraphs of this chapter. Addiotionally, the table \ref{Cuts_variables} shows all the cuts that were performed on the data in order to reduce as much as possible the backgrounds. 

%tabla preselection values
\begin{table}[h]
\centering
\caption{Preselection criteria}
\label{preselection_table}
\begin{tabular}{|c|c|}
\hline
Variables                                                                & Value                 \\ \hline
$p_T(b-jets)$ \& $p_T(\tau)$                                            & \textgreater 20.0 GeV \\ \hline
$p_T(jets)$                                                             & \textgreater 30.0 GeV  \\ \hline
$\Delta R (\tau, e)$, $\Delta R (\tau, \mu)$ \& $\Delta R (\tau, jets)$ & \textgreater 0.3      \\ \hline
$P_T(VBF\_jet)$                                                          & \textgreater 50.0 GeV \\ \hline
$\eta(VBF\_jet)$                                                         & \textgreater5.0       \\ \hline
diJetMass                                                               & \textgreater100.0 GeV \\ \hline
\end{tabular}
\end{table}

%tabla cortes

\begin{table}[h]
\centering
\caption{Cuts on different variables}
\label{Cuts_variables}
\begin{tabular}{|c|c|}
\hline
Variables                & Values                \\ \hline
n(jets)                  & \textless 5.0         \\ \hline
n($\tau$)                & = 1                   \\ \hline
n(b-jets)                & = 0                   \\ \hline
$|\eta(\tau)|$           & \textless 2.1         \\ \hline
$\vec{E_T^{miss}}$       & \textgreater 20.0 MeV \\ \hline
n($p_T(jet) > 50.0$)     & $\geq 2$              \\ \hline
$\eta(jet_l)\eta(jet_s)$ & \textless0            \\ \hline
$\Delta \eta$            & \textgreater3.8       \\ \hline
diJetMass                & \textgreater500MeV    \\ \hline
\end{tabular}
\end{table}


