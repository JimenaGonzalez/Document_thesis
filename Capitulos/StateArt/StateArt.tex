\chapter{State of the Art} 

\section{Standard Model}

\section{Higgs Mechanism}

\section{Neutrinos in the Standard Model}

As it was mentioned earlier the SM does not explain the reason why the mass of neutrinos is smaller than the mass of the other fermions by a factor of almost $10^{-6}$. Moreover, it does not
provide an explanation to the fact that only left handed netrinos had been observed in nature. 
In this section we are going to work on possible solutions to these problems. \footnote{The detailed calculation is explain in \ref{apendice_neutrinos}}

\subsection{Dirac Mass}
The lagrangian of a free fermion is:

\begin{equation}
 L = \bar{\psi} \left( i \gamma ^\mu \partial_{\mu} - m \right) \psi
\end{equation}

Where $\psi$ is the Dirac Spinor. The mass is included in the SM through the second term in the former equation, it is called ``Dirac mass term'':

\begin{equation}
 m \bar{\psi} \psi
\end{equation}

We can write the Dirac Spinor as a sum of it's left- and right- chiral states:

\begin{equation}
 m \bar{\psi} \psi = m \left( \overbar{\psi_L + \psi_R} \right) \left( \psi_L + \psi_R \right) = m \bar{\psi_L} \psi_R + m \bar{\psi_R}\psi_L
\end{equation}
Previously we have used the fact that: $\bar{\psi_L}\psi_L = \bar{\psi_R}\psi_R = 0$ which is proved in \ref{apendice_neutrinos}. It can be seen from the lastest equation
that a massive particle must have both quiral states: left and right. Thus, the Dirac Mass can be interpreted as the coupling constant between the two chiral states. Since right-handed 
neutrinos had never been observed in nature, it is expected that neutrinos have zero mass. Although the experiments of neutrino oscillations indicate that neutrinos have a small mass of the order of meV. The former implies either the existence of a right-handed neutrino which is responsable for the mass of the neutrino or there other sort of mass term.

\subsection{Majorana Mass}


The Majorana mechanism is based in the reasoning of writing the mass term in the Lagrangian only in term of the left-handed chiral state. We start by decomposing the wavefunction into its left and right chiral states in the Dirac Lagrangian: 


\begin{align}
  \phantom{i = j = k}
  &\begin{aligned}
    \mathllap{L} &= \bar{\psi} \left( i \gamma ^\mu \partial_{\mu} - m \right) \psi \\
    \mathllap{}  &= (\bar{\psi_L} + \bar{\psi_R})( i \gamma ^\mu \partial_{\mu} - m)(\psi_L + \psi_R) \\
    \mathllap{}  &= i \bar{\psi_L} \gamma ^\mu \partial_{\mu} \psi_L + i \bar{\psi_L} \gamma ^\mu \partial_{\mu} \psi_R - m \bar{\psi_L} \psi_L - m \bar{\psi_L} \psi_R \\
  &\qquad + i \bar{\psi_R} \gamma ^\mu \partial_{\mu} \psi_L + i \bar{\psi_R} \gamma ^\mu \partial_{\mu} \psi_R - m \bar{\psi_R} \psi_L - m \bar{\psi_R} \psi_R \\
  \end{aligned}\\
  &\begin{aligned}
   \mathllap{} &= i \bar{\psi_L} \gamma ^\mu \partial_{\mu} \psi_L - m \bar{\psi_L} \psi_R - m \bar{\psi_R} \psi_L + i \bar{\psi_R} \gamma ^\mu \partial_{\mu} \psi_R \\
   \end{aligned}
\end{align}


\section{Seesaw Mechanism}

