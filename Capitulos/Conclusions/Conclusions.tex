\chapter{Conclusions} 
\label{Conclusion_chapter}

The objetive of this work was to perfom a phenomenological analysis to determine a study that could reduce the backgrounds of our signal of interest under it. In order to do this, we started in Chapter \ref{State_Art_chapter} by making a brief summary of the SM and of what it states for neutrinos. Then, we studied the Majorana idea of writing the right-handed field in terms of the left-handed field. The former leads to the description of a Majorana mass term and the definition of a Majorana particle. Then, it is analized the Seesaw mechanism and the explanations it has for the mass of neutrinos are discussed. After, in Chapter \ref{Important_concepts_chapter} the relevant concepts and kinematical variables for this analysis were defined and explained.

In Chapter \ref{CMS_chapter} the different parts of the CMS detector are described with the explanation of how they work and their characteristics. The CMS detector was explained with detail because in this analysis we considered events generated at this detector. Next, in Chapter \ref{Model_chapter} the topology of the signals are explained: it is expected that the product tau has a displaced vertex, the presence of high energetic jets related to the VBF process, among other characteristics. The backgrounds of these signals were also explained with their final state characteristics. In Chapter \ref{Methodology_chapter} the computational tools that were used in this analysis are described. It was mentioned each software and their specific task on the simulation of the signal or analysis of the data.  

Then, the preselection criteria and the different cuts that were applied in this analysis were stated and explained in Chapter \ref{Event_selection_criteria_chapter}. The following Chapter \ref{Analysis_chapter} showed the analysis that was perfomed to study the signal and its background. It was showed that it was not possible to find optimal cuts in order to reduce the amount of background to distinguish signal. The former was because, as it was shown, the kinematical and topological distributions of the signals and backgrounds are very similar.
Additionally, the obtained values of the impact parameter for the signals were unexpected, because they were very low and it was expected they had a large impact parameter value. This feature can be a consequence that the quantity of signal was small and it is hypothesized that there can be a problem at the Delphes level of simulation at reconstructing secondary vertices.


