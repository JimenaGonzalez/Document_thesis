\chapter{Conclusions} 
\label{Conclusion_chapter}

The main objective of this work was to perform a phenomenological analysis to determine variables that could reduce the background, to be able to distinguish an event with the presence of a heavy neutrino. In order to do this, we started in Chapter \ref{State_Art_chapter} by making a brief summary of the SM and what it states for neutrino mass. Then, we studied the Majorana idea of writing the right-handed field in terms of its left-handed field. The former leads to the description of a Majorana mass term and the definition of a Majorana particle. Then, the Seesaw mechanism and its explanation for the neutrinos mass is discussed. In Chapter \ref{Important_concepts_chapter}, the relevant concepts and kinematic variables for this analysis were defined and explained.

In Chapter \ref{CMS_chapter}, the different parts of the CMS detector are described with the explanation of how they work and their characteristics. The CMS detector was explained with detail since we considered events produced in this detector. Next, in Chapter \ref{Model_chapter}, it is stated what is expected for the topology of the signals: the presence of a tau with displaced vertex, the presence of high energetic jets related to the VBF process, among other characteristics. The backgrounds of these signals were also explained with their final state characteristics. In Chapter \ref{Methodology_chapter}, the computational tools that were used in this analysis were described. It was mentioned each software and their specific task on the simulation of the signal and analysis of the data.  

Then, the preselection criteria and the different cuts that were applied in this analysis were stated and explained in Chapter \ref{Event_selection_criteria_chapter}. Chapter \ref{Analysis_chapter} showed the analysis that was performed to study the signal and its background. It was showed that it was not possible to find optimal cuts in order to reduce the amount of background to distinguish the signal. The former was because the topological distributions of the signal and backgrounds were very similar. This can be caused because the Z boson and the Higgs boson have a similar value of mass. Additionally, the obtained values of the impact parameter for the signals were unexpected, because they all were positive and very small. The former could indicate that there is a problem in the simulation of the signal, since it is expected that the impact parameter has a symmetric distribution.


