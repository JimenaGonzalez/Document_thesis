\chapter{Introduction}
\label{Introduction_chapter}


The Standard Model (SM) is a theory that collects our knowledge about the elementary particles and their interactions. Despite this theory is capable of explaining several physical phenomena that have been observed in experiments, there are some questions that this model does not answer. Thus, this theory is not complete. For example, several cosmological observations suggest the existence a new type of matter that is stable and that does not interact electromagnetically, referred to as dark matter (DM). Unfortunately, the SM does not provide a particle that fulfills the required characteristics of DM. Other dilemma the SM has is related to the mass of neutrinos: it predicts that the neutrino mass is zero. Nevertheless, the former is incorrect because the observation of neutrinos oscillations in multiple experiments demonstrates that neutrinos have mass \cite{Neutrino experiment 1 mass, Neutrino experiment 2 mass}. One additional fact that is very interesting and the SM can not explain is the observation of neutrinos with only left-handed helicity. 

This monograph focuses in models that propose the existence of heavy-neutrinos with right-handed helicity, which in some cases are postulated as DM candidates \cite{Neutrino dark matter candidate 1, Neutrino dark matter candidate 2}. Additionally, these models propose a mechanism by which neutrinos acquire mass. The Seesaw mechanism is a popular extension of the SM that includes heavy right-handed neutrinos. If the existence of these particles is proved, not only the helicity symmetry of neutrinos would be restored, but also it would explain how they gain mass. The search of these particles has been performed in the experiments LEP \cite{Lep experiment}, CMS \cite{CMS experiment} and ATLAS \cite{ATLAS experiment} without successful results yet.

Recently, it has been proposed a new mechanism of production of heavy neutrinos through the decay of the Higgs Boson \cite{Seesaw Mechanism with displaced vertices} using the Type I Seesaw mechanism. If the heavy neutrino mass is of the order of a few GeV, the Higgs boson would travel a certain distance before decaying. As a consequence, the decay products are expected to have associated tracks with displaced vertices. In this case the presence of tracks with displaced vertices in the detector is an important signal to prove the Seesaw mechanism. Nevertheless, due to experimental restrictions of the available triggers in CMS and ATLAS, the theoretical analysis proposed in reference \cite{Seesaw Mechanism with displaced vertices} is not achievable. This project proposes the production mechanism of the Higgs boson through Vector Boson Fusion (VBF), instead of gluon annihilation (Drell-Yan). The VBF jets in the event topology gives a new handle in order to trigger on these hypothetical signal events.

The observation of the Higgs decay into heavy neutrinos would be a firm proof of the Type I Seesaw mechanism \cite{Type I Seesaw Mechanism}, which would indicate the existence of physics beyond the SM. The Type I Seesaw mechanism is the simplest extension of the SM that is capable of explaining the smallness of the left-handed neutrino masses with respect to other fundamental particles. 

The main problem of detecting this event of interest is that the magnitude of its signal is significantly small with respect to other processes from the SM. For this reason, the processes from the SM that have the same or similar final states as the signal of interest are called backgrounds. Therefore, it is fundamental to develop procedures with the objective of reducing the experimental backgrounds under the magnitude of the searched signal. These procedures use different variables that exploit the topology of the event and its kinematic characteristics. When a set of variables that potentially separate the signal from the background are determined, it is necessary to find the optimal values of these variables that allow to reduce as much as possible the background. Generally, the optimization studies use figures of significance, such as: 

\begin{equation}
    \frac{S}{\sqrt{S+B}},
\end{equation}

where S and B represent the expect number of signal and background events respectively. 

This document is organized as follows. In Chapter \ref{State_Art_chapter} the state of the art for this study is stated: the SM is described and the Seesaw mechanism is explained. Then, in Chapter \ref{Important_concepts_chapter} the important concepts for this analysis, such as 
jets, luminosity, cross section, etc, are described. Additionally, the kinematical variables used in this analysis that have the potential of reducing the levels of background are defined and illustrated in this chapter. The simulation of the interaction of particles with a detector, was performed using the emulation of the CMS detector. The description of the CMS detector is found in Chapter \ref{CMS_chapter}. In this chapter there is also a brief description of the triggers performed at the CMS. In Chapter \ref{Model_chapter} there is an explanation of the topology of the signal and its possible final states. In Chapter \ref{Methodology_chapter} there is a description of the software tools used in this analysis. The computational programs used were: MadGraph which makes a simulation of the event, Pythia which simulates the processes of hadronization of the signal, Delphes which simulates the behaviour of a multipurpose detector, and ROOT that is used to perform the analysis of the signal and backgrounds. The pre-selection criteria and required cuts used in the study, with the corresponding motivation, are explained in Chapter \ref{Event_selection_criteria_chapter}.
In Chapter \ref{Analysis_chapter}, the analysis technique used for this study is described, and it is showed the performance of different variables and their potential to reduce the background is discussed. Finally, in Chapter \ref{Conclusion_chapter} the conclusions of this project are stated. 
