\chapter{Introduction}
\label{Introduction_chapter}

The Standard Model (SM) is a theory that collects our knowledge about the elementary particles and their interactions. Despite this theory is capable of explaining several physical phenomena that have been observed in experiments, there are some questions that this model do not answer. Thus, this theory is not complete. One example of the physical phenomena the SM does not explain are cosmological observations that suggest the existence of a new type of matter that is stable, massive and that does not interact with electromagnetic radiation, whereby it is called Dark Matter (DM). There is a disagreement between these astronomical observations and the theoretical predictions for the rotation velocity of stars and galaxies. For this reason, DM has been proposed. Unfortunately, the SM does not provide a particle that fulfills the required characteristics of DM. Other dilemma the SM has is related to the mass of neutrinos, the model predicts that it is zero. Nevertheless, the former is incorrect because the observations of neutrinos oscillations in multiple experiments demonstrate that neutrinos have mass \cite{Neutrino experiment 1 mass, Neutrino experiment 2 mass}. One additional fact that is very interesting and the SM can not explain is the observation of neutrinos with only left-handed helicity. . 

Since the SM does not explain different phenomena, in particular the observation of only left-handed neutrinos, some models that extend the SM have been constructed. Some of these models propose the existence of heavy neutrinos with right helicity, which in some cases are postulated as candidates of DM \cite{Neutrino dark matter candidate 1, Neutrino dark matter candidate 2}. Additionally, these models propose a mechanism by which neutrinos acquire mass. One of the most famous extension models of the SM, that propose the existence of heavy right-handed neutrinos, is the Seesaw Mechanism. If the existence of these particles is proved, not only the helicity symmetry of neutrinos would be restored, but also it would explain how them gain mass. The search of these particles has been perform in the experiments LEP \cite{Lep experiment}, CMS \cite{CMS experiment} and ATLAS \cite{ATLAS experiment} without successful results.

Recently, it has been proposed a new mechanism of production of heavy neutrinos through the decay of the Higgs Boson \cite{Seesaw Mechanism with displaced vertices} using the type I Seesaw mechanism. If the heavy neutrino mass is of the order of a few GeV the Higgs boson would travel a certain distance before decaying. As a consequence, the decay products are expected to have associated tracks with displaced vertices. For this reason, the presence of displaced vertices in the detector is an important signal to prove the model. Nevertheless, due to experimental restrictions of the available triggers in CMS and ATLAS, the theoretical analysis proposed in reference \cite{Seesaw Mechanism with displace vertices} is not achievable. Thus, in this project is proposed the search of the heavy neutrino when the Higgs boson is produced by a process denominated Vector Boson Fusion (VBF).

The observation of the Higgs decay into heavy neutrinos would be a firm prove of the Type I Seesaw mechanism \cite{Type I Seesaw Mechanism}, which would indicate the existence of physics beyond the SM associated to the mass of the neutrinos. The Type I Seesaw mechanism is the simplest extension of the SM that is capable of explaining the smallness of the neutrino with respect to other fundamental particles. 

The main problem of detecting an event of interest is that the magnitude of its signal is significantly smaller with respect to processes from the SM. For this reason, the processes from the SM that have the same or similar final states as the signal are called backgrounds. For this reason it is fundamental to develop procedures in order to reduce the experimental backgrounds under the magnitude of the searched signal. These procedures usually use different variables that exploit the topology of the event and its kinematic characteristics. When a set of variables that potentially separate the signal from the background is determined, it is necessary to find the optimal values of them that allow to reduce as much as possible the background. Generally, the optimization studies use figures of significance, such as: 

\begin{equation}
    \frac{S}{\sqrt{S+B}},
\end{equation}

where S and B represents the expected event number of signal and background correspondingly.

This document is organized as follows. In \ref{State_Art_chapter} chapter the State of the Art for this study is stated: the SM is described and the Seesaw mechanism is explained. Then, in the chapter \ref{Important_concepts_chapter} the important concepts for this analysis, such as jet and cross section, are described. Additionally, the kinematical variables used in this analysis that have the potential of reducing the levels of backgrounds are defined and illustrated. Next, in the chapter \ref{CMS_chapter} there is information about the detector CMS, where is described how each of its parts functions and their specific tasks in the detector. The CMS detector is conformed by the tracking system, the electromagnetic and hadron calorimeters and the muon systems. There is also a brief description of the triggers performed at the CMS. In the chapter \ref{Model_chapter} there is an exposition of the event of interest, it is described the topology of the signal and its possible final states. Then, the backgrounds for this signal are mentioned with their corresponding final states. In chapter \ref{Methodology_chapter} there is a description of the software tools used in this analysis. The computational programs used are: MadGraph which makes a simulation of the event, Pythia which simulates the processes of hadronization of the signal, Delphes which simulates the behaviur of a multipurpuose detector, and ROOT that is used to perform the analysis of the signal and backgrounds. In the next chapter, \ref{Event_selection_criteria_chapter}, the different preselection values imposed are enumerated. Then, all the required cuts on the variables are described with an explanation of why they are done with the corresponding values. In chapter \ref{Analysis_chapter} there is information about the analysis that was performed, it shows the performance of different variables and their potential to reduce the backgrounds is discussed. Finally, at chapter \ref{Conclusion_chapter} the conclusions of this project are stated. 
