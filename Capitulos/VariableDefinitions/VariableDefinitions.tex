 \chapter{Important Concepts and Variable Definitions}
 
 \section{Jets}
 A Jet can be defined as a high energy shower of stable particles that comes from fragmentation of quarks or gluons. The initial quarks and gluons in the process are know ``initial partons''.
 Due to the initial partons are colour charged, they can not be isolated singularly (this phenomenon is called ``colour confinement''. Since it is not possible for charged particles to be isolated 
 they must go through a non-perturvative process that converts them into colour neutral particles. This process is called ``hadronization'' and there are different models to explain it. According to 
 the string model, the confining nature of strong interaction increases the potential colour in a proportional way as the distance between the initial partons. When the distance reaches a certain 
 critical value it is energetically favourable to produce a quark pair from the vacuum. Finally, by this proccess the inital colour charged particles are convert into bound colour-singlet hadronic 
 states. \\
 
 Besides jets may display a structure with properties that could indicate which were the initial partons interacting, they are hard to study individually when there is a numerous quantity of them
 in an event. The former is because it is almost imposible to attach all particles in an event final state to a single initial parton. The reconstruction of jets depends of elements like the 
 fragmentation process, detectors effects, among others. Thus, there exist algorithms which cluster some particles in a final state so that it is possible to determine properties as 4-momentum 
 and jet shapes. The objetive of this algorithms is to determine the inital interacting partons and approximate its directions and energies. \\
 
 According to the reconstruction algorithms we can define a jet at three different levels.
 
 
 \section{Cross Section and Luminosity}
 
 \section{Pseudorapidity}
 
 \subsection{Coordinate System of CMS and ATLAS dectector at the LHC}
 
 \section{Minimal Separation Distance Between Particles}
 
 \section{MET}

 
 \section{Impact Parameter}

 
 