\chapter{CMS Detector}

The CMS is one of the four experiments at the Large Hadron Collider (LHC), which is the largest and most powerfull particle accelerator in the world. This accelerator collitionates protons at very high energies of the order of 13 TeV, with the objetive of studying the elemental particles of the universe. 
The LHC is conform by a ring of almost 27 km and by 4 detectors located at the different collition points of the ring. 2 of these detector are general-purpose, they belong to the experiments CMS (Compact Muon Solenoid) and ATLAS (A Toroidal LHC ApparatuS). Both experiments share the same goal of searching physics beyond the SM as: search and measurement of the Higgs boson, Supersymmetry searches, DM, detection of extra dimensions, among others. The difference between both experiments is that they use different technical solutions and the magnetic field is produced by a different system design. In the ATLAS detector the magnetic field is produced by a central toroid, two end toroids and a central solenoid, while the CMS detector is built around a solenoid magnet.  

The CMS and ATLAS dectectors have a cylindrical form in order to have the most uniform magnetic field posible, they are centered in the direction of the interacting beams and the collition point and have two ``end-caps'' to cover the forward regions. These detectors are conformed by the same general components, from the inner part of the detector to the outer part, these are: a tracking system, an electromagnetic and a hadronic calorimeter and muon detectors. They also have magnets to curve the path of the electric charged particles, so it can be determined if a particle has a positive or negative charge.  

\section{Accelarator and Storage Ring}


\section{Tracks Detector}

Since every 25 ns almost 1000 particles are going to be produce in the center of the CMS detector, it is necessary to have a tracking system that convers this region. This tracking system must measure the momentum and vertices of the particles with a high precision. The tracking system is based on silicon detectors, with high granularity pixel systems at the smallest radii, and silicon-strip detectors at larger ones. 
%Citar libro:perspectives on LHC Physics
The elements of the tracking system are fast taken measurements, tolerate high doses of radiation, are made of ligh material and resist the severe conditions imposed by low temperature. One of the major challenges for the inner dectector parts is the control of aging effects because the damage produced by irradiation is severe. The silicon detectors are p-n junction diodes, so when a particle crosses the detector liberates electron-hole pairs, which move to the electrodes of the system. The tracking system in the CMS detector covers the range of $| \eta|<2.5$, the region where most of the particles arrive. 
The flux of particles arriving to the detector depends of the distance from the center of collition: as the flux crosses the detector the quantity of particles in it decrease. Thus, the resolution of the tracking system does not need to be so high in the intermediate and end caps of if. For this reason, in the first region of the tracking system (the closest to the interaction point), there are silicon pixel detectors with cell size of 100 $\times$ 150 $\mu m^2$. Due to these silicon pixels are expensive and have high power density at the intermediate region of the tracking detector they are replaced by silicon microstrip systems.

\section{Hadronic Calorimeter}

\section{Electromagnetic Calorimeter}

\section{Muon Detector}

\section{Triggers}


\subsection{Triggers at the CMS}