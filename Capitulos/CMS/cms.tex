\chapter{CMS Detector}

The CMS is one of the four experiments at the Large Hadron Collider (LHC), which is the largest and most powerfull particle accelerator in the world. This accelerator collitionates protons at very high energies of the order of 13 TeV, with the objetive of studying the elemental particles of the universe. 
The LHC is conform by a ring of almost 27 km and by 4 detectors located at the different collition points of the ring. 2 of these detector are general-purpose, they belong to the experiments CMS (Compact Muon Solenoid) and ATLAS (A Toroidal LHC ApparatuS). Both experiments share the same goal of searching physics beyond the SM as: search and measurement of the Higgs boson, Supersymmetry searches, DM, detection of extra dimensions, among others. The difference between both experiments is that they use different technical solutions and the magnetic field is produced by a different system design. In the ATLAS detector the magnetic field is produced by a central toroid, two end toroids and a central solenoid, while the CMS detector is built around a solenoid magnet.  

The CMS and ATLAS dectectors have a cylindrical form in order to have the most uniform magnetic field posible, they are centered in the direction of the interacting beams and the collition point and have two ``end-caps'' to cover the forward regions. These detectors are conformed by the same general components, from the inner part of the detector to the outer part, these are: a tracking system, an electromagnetic and a hadronic calorimeter and muon detectors. They also have magnets to curve the path of the electric charged particles, so it can be determined if a particle has a positive or negative charge. The measurement of the curve can be used to calculate the momentum of the charged particle.

\section{Accelarator and Storage Ring}


\section{Tracks Detector}

Since every 25 ns almost 1000 particles are going to be produce in the center of the CMS detector, it is necessary to have a tracking system that convers this region. This tracking system must measure the momentum and vertices of the particles with a high precision. The tracking system is based on silicon detectors, with high granularity pixel systems at the smallest radii, and silicon-strip detectors at larger ones. 
%Citar libro:perspectives on LHC Physics
The elements of the tracking system are fast taken measurements, tolerate high doses of radiation, are made of ligh material and resist the severe conditions imposed by low temperature. One of the major challenges for the inner dectector parts is the control of aging effects because the damage produced by irradiation is severe. The silicon detectors are p-n junction diodes, so when a particle crosses the detector liberates electron-hole pairs, which move to the electrodes of the system. The tracking system in the CMS detector covers the range of $| \eta|<2.5$, the region where most of the particles arrive. 

The flux of particles arriving to the detector depends of the distance from the center of collition: as the flux crosses the detector the quantity of particles in it decrease. Thus, the resolution of the tracking system does not need to be so high in the intermediate and end caps of if. For this reason, in the first region of the tracking system (the closest to the interaction point), there are silicon pixel detectors with cell size of 100 $\times$ 150 $\mu m^2$. THe innermost layer of pixels is located as near to the beam as it is practical, this is at a radius around 4.5 cm.

Due to the silicon pixels are expensive and have high power density, at the intermediate region of the tracking detector they are replaced by silicon microstrip systems. The former is also because at that region (at a radius between 20 and 55 cm) the flux of particles is low enough to use this technology. These barrel cylinders and end-caps dics, as the silicon pixels ,cover the region of $|\eta| < 2.5$ . The strip dimensions are or around $11 cm \times 100 \mu m$. These silicon microstrip are arranged in a special way to improve the resolution in the z axis. 

In the outermost region of the tracking system (at a radius greater that 55 cm) the particle flux is low enough to use a larger-pitch silicon microstrips. The maximum size of these cells is $25cm \times 80 \mu m$. There are 6 layers of this silicon microstrips moduls in the barrel and 9 end-caps discs that also cover the region given by $|\eta|< 2.5$.

%Escribir interaccion particulas con tracking systems

\section{Calorimetry}

Surrounding the tracking system of the CMS detector is located the electromagnetic and hadronic calorimeter. The calorimeters measure the energy of the incoming particles, by absorving it and transforming it into heat. The priority of the electromagnetic calorimeter is to measure precisely the energy of electrons and photons, to make measurements of their position and direction of movement. While, the priority of the hadronic calorimeter is to make precise measurements of the jets energy and to cover a larger area of $|\eta| < 5$, for the purpose of attributte all the $\vec{E}_T^{miss}$ to the actuall non-detected particles. 

The electromagnetic and hadron calorimeter are made out of scintillation crystals. When a high energy particle goes through the detector, it collides with the nuclei of the material and generates a shower of particles. The product particles of this interaction excitate the atoms in the material by making the electrons in them go to a higher orbit. When each electron returns to the initial orbit, it emmits a photon. 

Then, the ligh emmited by the scintillator is measured by photodiodes  

\subsection{Electromagnetic Calorimeter}

The electromagnetic calorimeter is an entirely active homogeneus calorimeter made of lead tungstate (PbWO$_4$) crystal. It has 61,200 crystal in the central barrel part and 7,324  in each of the two end-caps. As a consequence of the use of high density crystals the calorimeter is fast, has fine granularity and is radiation resistant. The lead tungstate crystal material was chosen for different reasons. First, it emmits a short radiation lenght when a particle ionize the material by crossing the crystal. Second, it has small Moliere radius, which is defined as the radius of the cylinder surrouding the 90\% of the shower's energy deposition. The former leads to a compact calorimeter in size. Third, the lead tungstate crystal has short decay time constant , which allow the calorimeter to have a fast response. Lastly, it is resistant to high dosis of radiation. 

The electromagnetic calorimeter is located within the solenoid. The light emmited by the 


\subsection{Hadronic Calorimeter}

\section{Muon Detector}

\section{Triggers}


\subsection{Triggers at the CMS}




