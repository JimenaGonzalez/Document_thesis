%Plantilla basada en "Template for Masters / Doctoral Thesis" (plantilla disponible en writeLaTex) que subió LaTeXTemplates.com

%\documentclass[11pt]{book}
\documentclass[11pt, oneside]{book}
\usepackage[paperwidth=17cm, paperheight=22.5cm, bottom=2.5cm, right=2.5cm]{geometry}
\usepackage{amssymb,amsmath,amsthm} %paquete para símbolo matemáticos
%\usepackage[spanish]{babel}
\usepackage[english]{babel}
\usepackage[utf8]{inputenc} %Paquete para escribir acentos y otros símbolos directamente
\usepackage{enumerate}
\usepackage{graphicx}
\usepackage[titletoc]{appendix}
\usepackage{mathtools}
\usepackage{caption}
\usepackage{subcaption}
%\usepackage{subfig} %para poner subfiguras
\graphicspath{{Img/}} %En qué carpeta están las imágenes
%\usepackage[nottoc]{tocbibind}
\usepackage[pdftex,
            pdfauthor={SANDRA JIMENA GONZÁLEZ LOZANO},
            pdftitle={Phenomenological Study of Search of Heavy Neutrinos, with Displaced Vertices and Vector Boson Fusion},
            pdfsubject={High Energy Physics},
            pdfkeywords={Heavy neutrino, },
            pdfproducer={Latex con hyperref},
            pdfcreator={pdflatex}]{hyperref}
\newcommand{\overbar}[1]{\mkern 1.5mu\overline{\mkern-1.5mu#1\mkern-1.5mu}\mkern 1.5mu}


\begin{document}

%\graphicspath{./Capitulos/VariableDefinitions}
%----------------------------------------------------------------------------------------
%	COMANDOS PERSONALIZADOS
%----------------------------------------------------------------------------------------

%SI TU TESIS TIENE TEOREMAS Y DEMOSTRACIONES, PUEDES DESCOMENTAR Y USAR LOS SIGUIENTES COMANDOS

%\renewcommand{\proofname}{Demostración}
%\providecommand{\norm}[1]{\lVert#1\rVert} %Provee el comando para producir una norma.
%\providecommand{\innp}[1]{\langle#1\rangle} 
%\newcommand{\seno}{\mathrm{sen}}
%\newcommand{\diff}{\mathrm{d}}

%\newtheorem{teo}{Teorema}[section] 
%\newtheorem{cor}[teo]{Corolario}
%\newtheorem{lem}[teo]{Lema}

%\theoremstyle{definition}
%\newtheorem{dfn}[teo]{Definición}

%\theoremstyle{remark}
%\newtheorem{obs}[teo]{Observación}

%\allowdisplaybreaks


%----------------------------------------------------------------------------------------
%	PORTADA
%----------------------------------------------------------------------------------------

\title{Document_PhenoAnalysis} %Con este nombre se guardará el proyecto en writeLaTex

\begin{titlepage}
\begin{center}

\textsc{\Large Universidad de los Andes}\\[1em]

%Figura
\begin{figure}[h]
\begin{center}
\includegraphics[scale=0.4]{logo_uniandes.png}
\end{center}
\end{figure}

\vspace{4em}

\textsc{\huge \textbf{Phenomenological Study of Search of Heavy Neutrinos, with Displaced Vertices and Vector Boson Fusion}}\\[4em]

%\textsc{\large Tesis}\\[1em]

\textsc{This dissertation is submitted for the degree of}\\[1em]

\textsc{Physicist}\\[1em]

\textsc{by}\\[1em]

\textsc{\Large Sandra Jimena González Lozano}\\[1em]

\textsc{\large Advisor: Andrés Flórez}

\end{center}

\vspace*{\fill}
\textsc{Bogotá, D.C. \hspace*{\fill} 2017}

\end{titlepage}


%----------------------------------------------------------------------------------------
%	DECLARACIÓN
%----------------------------------------------------------------------------------------

%\thispagestyle{empty}
%\vspace*{\fill}
%\begingroup
%``Con fundamento en los artículos 21 y 27 de la Ley Federal del Derecho de Autor y como titular de los derechos moral y patrimonial de la obra titulada ``\textbf{TÍTULO DE LA TESIS}'', otorgo de manera gratuita y permanente al Instituto Tecnológico Autónomo de México y a la Biblioteca Raúl Bailléres Jr., la autorización para que fijen la obra en cualquier medio, incluido el electrónico, y la divulguen entre sus usuarios, profesores, estudiantes o terceras personas, sin que pueda percibir por tal divulgación una contraprestación''.
%\thispagestyle{empty}
%\vspace*{\fill}
%\begingroup
%``Con fundamento en los artículos 21 y 27 de la Ley Federal del Derecho de Autor y como titular de los derechos moral y patrimonial de la obra titulada ``\textbf{TÍTULO DE LA TESIS}'', otorgo de manera gratuita y permanente al Instituto Tecnológico Autónomo de México y a la Biblioteca Raúl Bailléres Jr., la autorización para que fijen la obra en cualquier medio, incluido el electrónico, y la divulguen entre sus usuarios, profesores, estudiantes o terceras personas, sin que pueda percibir por tal divulgación una contraprestación''.

%\centering

%\hspace{3em}
%Prefacio}
%\textsc{AUTOR}

%\vspace{5em}

%\rule[1em]{20em}{0.5pt} % Línea para la fecha

%\textsc{Fecha}
 
%\vspace{8em}

%\rule[1em]{20em}{0.5pt} % Línea para la firma

%\textsc{Firma}

%\endgroup
%\vspace*{\fill}
%\centering

%\hspace{3em}

%\textsc{AUTOR}

%\vspace{5em}

%\rule[1em]{20em}{0.5pt} % Línea para la fecha

%\textsc{Fecha}
 
%\vspace{8em}

%\rule[1em]{20em}{0.5pt} % Línea para la firma

%\textsc{Firma}

%\endgroup
%\vspace*{\fill}


%----------------------------------------------------------------------------------------
%	DEDICATORIA
%----------------------------------------------------------------------------------------

%\pagestyle{empty}
%\frontmatter

%\chapter*{}
%\begin{flushright}
%\textit{DEDICATORIA}
%\end{flushright}


%----------------------------------------------------------------------------------------
%	AGRADECIMIENTOS
%----------------------------------------------------------------------------------------

%\chapter*{Agradecimientos}
%\markboth{AGRADECIMIENTOS23}{AGRADECIMIENTOS} % encabezado 

%¡Muchas gracias a todos!


%----------------------------------------------------------------------------------------
%	PREFACIO
%----------------------------------------------------------------------------------------

%\chapter*{Prefacio}

%\pagestyle{plain}
%\markboth{PREFACIO23}{PREFACIO} % encabezado 

%PUEDEN QUITAR ESTA PARTE


%----------------------------------------------------------------------------------------
%	TABLA DE CONTENIDOS
%---------------------------------------------------------------------------------------

\tableofcontents
\listoffigures

%----------------------------------------------------------------------------------------
%	TESIS
%----------------------------------------------------------------------------------------
\mainmatter %empieza la numeración de las páginas
\pagestyle{headings}

%  Incluye los capítulos en el folder de capítulos

\chapter{Introduction}
\label{Introduction_chapter}


The Standard Model (SM) is a theory that collects our knowledge about the elementary particles and their interactions. Despite this theory is capable of explaining several physical phenomena that have been observed in experiments, there are some questions that this model does not answer. Thus, this theory is not complete. For example, several cosmological observations suggest the existence a new type of matter that is stable and that does not interact electromagnetically, referred to as dark matter (DM). Unfortunately, the SM does not provide a particle that fulfills the required characteristics of DM. Other dilemma the SM has is related to the mass of neutrinos: it predicts that the neutrino mass is zero. Nevertheless, the former is incorrect because the observation of neutrinos oscillations in multiple experiments demonstrates that neutrinos have mass \cite{Neutrino experiment 1 mass, Neutrino experiment 2 mass}. One additional fact that is very interesting and the SM can not explain is the observation of neutrinos with only left-handed helicity. 

This monograph focuses in models that propose the existence of heavy-neutrinos with right-handed helicity, which in some cases are postulated as DM candidates \cite{Neutrino dark matter candidate 1, Neutrino dark matter candidate 2}. Additionally, these models propose a mechanism by which neutrinos acquire mass. The Seesaw mechanism is a popular extension of the SM that includes heavy right-handed neutrinos. If the existence of these particles is proved, not only the helicity symmetry of neutrinos would be restored, but also it would explain how they gain mass. The search of these particles has been performed in the experiments LEP \cite{Lep experiment}, CMS \cite{CMS experiment} and ATLAS \cite{ATLAS experiment} without successful results yet.

Recently, it has been proposed a new mechanism of production of heavy neutrinos through the decay of the Higgs Boson \cite{Seesaw Mechanism with displaced vertices} using the Type I Seesaw mechanism. If the heavy neutrino mass is of the order of a few GeV, the Higgs boson would travel a certain distance before decaying. As a consequence, the decay products are expected to have associated tracks with displaced vertices. In this case the presence of tracks with displaced vertices in the detector is an important signal to prove the Seesaw mechanism. Nevertheless, due to experimental restrictions of the available triggers in CMS and ATLAS, the theoretical analysis proposed in reference \cite{Seesaw Mechanism with displaced vertices} is not achievable. This project proposes the production mechanism of the Higgs boson through Vector Boson Fusion (VBF), instead of gluon annihilation (Drell-Yan). The VBF jets in the event topology gives a new handle in order to trigger on these hypothetical signal events.

The observation of the Higgs decay into heavy neutrinos would be a firm proof of the Type I Seesaw mechanism \cite{Type I Seesaw Mechanism}, which would indicate the existence of physics beyond the SM. The Type I Seesaw mechanism is the simplest extension of the SM that is capable of explaining the smallness of the left-handed neutrino masses with respect to other fundamental particles. 

The main problem of detecting this event of interest is that the magnitude of its signal is significantly small with respect to other processes from the SM. For this reason, the processes from the SM that have the same or similar final states as the signal of interest are called backgrounds. Therefore, it is fundamental to develop procedures with the objective of reducing the experimental backgrounds under the magnitude of the searched signal. These procedures use different variables that exploit the topology of the event and its kinematic characteristics. When a set of variables that potentially separate the signal from the background are determined, it is necessary to find the optimal values of these variables that allow to reduce as much as possible the background. Generally, the optimization studies use figures of significance, such as: 

\begin{equation}
    \frac{S}{\sqrt{S+B}},
\end{equation}

where S and B represent the expect number of signal and background events respectively. 

This document is organized as follows. In Chapter \ref{State_Art_chapter} the state of the art for this study is stated: the SM is described and the Seesaw mechanism is explained. Then, in Chapter \ref{Important_concepts_chapter} the important concepts for this analysis, such as 
jets, luminosity, cross section, etc, are described. Additionally, the kinematical variables used in this analysis that have the potential of reducing the levels of background are defined and illustrated in this chapter. The simulation of the interaction of particles with a detector, was performed using the emulation of the CMS detector. The description of the CMS detector is found in Chapter \ref{CMS_chapter}. In this chapter there is also a brief description of the triggers performed at the CMS. In Chapter \ref{Model_chapter} there is an explanation of the topology of the signal and its possible final states. In Chapter \ref{Methodology_chapter} there is a description of the software tools used in this analysis. The computational programs used were: MadGraph which makes a simulation of the event, Pythia which simulates the processes of hadronization of the signal, Delphes which simulates the behaviour of a multipurpose detector, and ROOT that is used to perform the analysis of the signal and backgrounds. The pre-selection criteria and required cuts used in the study, with the corresponding motivation, are explained in Chapter \ref{Event_selection_criteria_chapter}.
In Chapter \ref{Analysis_chapter}, the analysis technique used for this study is described, and it is showed the performance of different variables and their potential to reduce the background is discussed. Finally, in Chapter \ref{Conclusion_chapter} the conclusions of this project are stated. 

%\thispagestyle{empty}
\chapter{State of the Art} 

\section{Standard Model}

\section{Higgs Mechanism}

\section{Neutrinos in the Standard Model}

As it was mentioned earlier the SM does not explain the reason why the mass of neutrinos is smaller than the mass of the other fermions by a factor of almost $10^{-6}$. Moreover, it does not
provide an explanation to the fact that only left handed netrinos had been observed in nature. 
In this section we are going to work on possible solutions to these problems. \footnote{The detailed calculation is explain in \ref{apendice_neutrinos}}

\subsection{Dirac Mass}
The lagrangian of a free fermion is:

\begin{equation}
 L = \bar{\psi} \left( i \gamma ^\mu \partial_{\mu} - m \right) \psi
\end{equation}

Where $\psi$ is the Dirac Spinor. The mass is included in the SM through the second term in the former equation, it is called ``Dirac mass term'':

\begin{equation}
 m \bar{\psi} \psi
\end{equation}

We can write the Dirac Spinor as a sum of it's left- and right- chiral states:
%PIE DE PAGINA A APENDICE
\begin{equation}
 m \bar{\psi} \psi = m \left( \overbar{\psi_L + \psi_R} \right) \left( \psi_L + \psi_R \right) = m \bar{\psi_L} \psi_R + m \bar{\psi_R}\psi_L
\end{equation}
Previously we have used the fact that: $\bar{\psi_L}\psi_L = \bar{\psi_R}\psi_R = 0$ which is proved in \ref{apendice_neutrinos}. It can be seen from the lastest equation
that a massive particle must have both quiral states: left and right. Thus, the Dirac Mass can be interpreted as the coupling constant between the two chiral states. Since only left-handed 
neutrinos had been observed in nature, it is expected that neutrino has zero mass. Although the experiments of neutrino oscillations 




\section{Seesaw Mechanism}


%\thispagestyle{empty}
 \chapter{Important Concepts and Variable Definitions}
 \label{Important_concepts_chapter}
 
 \section{Jets}
 A jet can be defined as a high energy shower of stable particles that comes from fragmentation of quarks or gluons \cite{Particle_Detectors_Claus}. The initial quarks and gluons in the process are known as ``initial partons''. Due to that initial partons are colour charged, they can not be isolated singularly (this phenomenon is called ``colour confinement''). Since it is not possible for coloured particles to be isolated, they must go through a non-perturvative process that converts them into colour neutral particles. This process is called ``hadronization'', and there are different models to explain it. According to the string model, the confining nature of strong interaction increases the potential colour is proportional to the distance between the initial partons. When the distance reaches a certain critical value, it is energetically favourable to produce a quark pair from the vacuum. Finally, by this proccess the inital colour charged particles are converted into bound colour-singlet hadronic states.  
 %Ya citado libro: Particle detectors Claus Grupen and Boris shwartz
 
 Despite jets may display a structure with properties that could indicate which were the initial partons interacting, they are hard to study individually when there is a numerous quantity of them
 in an event. The former is because it is almost imposible to associate all particles in an event final state to a single initial parton. The reconstruction of jets depends on elements like the 
 fragmentation process, detectors effects, among others. Thus, there exist algorithms that cluster some particles in a final state so it is possible to determine properties as 4-momentum 
 and jet shapes. The objective of these algorithms is to determine the inital interacting partons and approximate its directions and energies.

According to the reconstruction algorithms, we can define a jet at three different levels. At parton level, a jet can be understood as a quark or a gluon. At hadronic level, can be referred to the hadrons produced due to the hadronization process, like kaons or pions. Finally, at a detector level, a jet can be understood as a set of reconstructed tracks spatially associated with energy 
 deposited in the calorimeters \cite{Tesis_luis_alfredo}. The reconstuction algorithms that are going to be used for this analysis consist in the use of mathematical cones that enclose the regions where a large quantity of particles are detected. The radius of the cone must be large enough to enclose all the particles coming from the initial quark or gluon, and must be small enough to not include other particles that belong to a different jet. The three level definitions just mentioned are illustrated in Figure \ref{Jets_definitions}. In this figure the mathematical cones used for the reconstruction of a jet are also showed.
 %Citar tesis Luis Alfredo
  
 \begin{figure}[h] 
 \centering
 \caption{Description of a jet at three different levels: partonic, hadronic and detector. Image taken from \cite{Image_jet_definitions}}
 \includegraphics[width=0.75\textwidth]{./Capitulos/VariableDefinitions/jets_definitions}  
 \label{Jets_definitions}
 \end{figure}
 
 \section{Cross Section and Luminosity}
 
 In High Energy Physics, the cross section $\sigma$ represents the probability that a given physical proccess occurs. This quantity is proportional to the energy of the event production. The unit 
 used for cross sections is the barn (1b = $10^{-28} \text{m}^2$). The number of expected events of a certain interaction in a fixed target experiment is proportional to its cross section, the particle flux, and the
 number of atoms per cubic meter in the target multiplied by the length \cite{Data_analysis_techniques}. The inverse of the number of atoms per area is called ``target constant $F$'' and it has the dimesion of an area. Thus, it 
 is possible to make an estimation of the number of interactions per second using:
 
 %citado libro data analysis techniques for high energy physics
 \begin{equation}
 \label{luminosity}
  \frac{N_{events}}{s} = \sigma \times \frac{N_{flux}/s}{F} = \sigma \times Luminosity
 \end{equation}

 In the Equation \ref{luminosity}, we defined the concept of luminosity, which depends on the beam energy and dynamics.  The luminosity is a quantity that is used to 
 describe the performance of a particle accelerator. It has units of the inverse of cross section, which is know as inverse barns $fb^{-1}$ and it is equivalent to $(1 fb = 10^{-28} m^2)$. 
 In particle colliders, the 
 luminosity depends on different variables such as the number of particles per bunch $N_b$, the number of bunches in each beam $\kappa_b$, the revolution frequency $f$ at the storage ring and the beam radii 
 $R$ of the bunches at the crossing point:
 
 \begin{equation}
  L = \frac{N_b^2 f \kappa_b}{4\pi R^2} 
 \end{equation}

 \section{Pseudorapidity}
 %GENERALIZAAAAAAAAAAAAAAAAAAAAAAAAAAAAAAAAAAAAAAAAAAAAAAAAAAAAAAAAR
  
The pseudorapidity is a variable that is defined in terms of the polar angle of the CMS and ATLAS dectector coordinates. These coordinates are ilustrated in the Figure \ref{CMSCoordinates}.
 
 \begin{figure}[h]
 \centering
 \caption{CMS and ATLAS detector coordinates. Image taken from \cite{CMS_ATLAS_coordinates}}
 \includegraphics[width=0.75\textwidth]{./Capitulos/VariableDefinitions/CMS_coordinates}  
 \label{CMSCoordinates}
 \end{figure}

The origin of the CMS and ATLAS coordinates coincides with the point in which a collision occurs in the detectors. 
The polar angle is described by the parameter $\theta$ and it is measured with respect to the z axis.
The azimuthal angle is denoted by $\Phi$ and it is measured in the xy plane from the x axis. The pseudorapidity is defined in terms of the polar angle as:

\begin{equation}
 \eta \equiv - \ln \left( \tan (\theta /2 ) \right)
\end{equation}

 The motivation to define and use this variable is that while $\Delta \theta$ is not a Lorentz invariant $\Delta \eta$ is. Moreover, the quantity of particles depending on the variable $\eta$
 is approximately uniform in a cilindrical detector. 
 
  
 \section{$p_T$, $\vec{E_T^{miss}}$ and $H_T$}

 The quantity $p_T$ is the transverse momentum, and it is the projection of the linear momentum onto the xy plane. This variable is used instead of the linear momentum because the initial beams
 are moving just in the z axis (the initial momentum in the xy plane is zero), so when a collision is produced the interesting effects occur in the transverse plane.
 
 As it was already mentioned, the momentum in the tranverse plane is zero before the collision. Since the tranverse momentum has to be conserved, after the collision it must also be zero. We can write
 the total momentum as the sum of the particles that are detected (visible particles) and the ones that are not detected (invisible particles), which can be expressed as:
 
 \begin{equation}
  0 = \sum_{i=1}^N \vec{P_T(i)} = \sum_{j=1}^M \vec{P_T(j)}^{visible} + \sum_{k=M}^{N-M} \vec{P_T(k)}^{invisible}
  \label{motivation_MET}
 \end{equation}

 The Equation \ref{motivation_MET} motivates the definition of a variable called ``Missing transverse energy'' ($\vec{E_T^{miss}}$), which is defined as the sum of the transverse momentum of the invisible
 particles:
 
 \begin{equation}
  \vec{E_T^{miss}} \equiv \sum_{k=M}^{N-M}\vec{P_T(k)}^{invisible} = - \sum_{j=1}^M  \vec{P_T(j)}^{visible}
 \end{equation}

Finally, the variable $H_T$ is defined as the sum of the transverse momentum of all the jets in the event, as it is showed in Equation

\begin{equation}
H_T = \sum_i^n P_T(jet_{\ i})
\end{equation}
 
 \section{Displaced Vertices and Impact Parameter}
The vertex of a track is a variable of importance because it can be used to determine the position of the point of interaction and the momentum vector of the tracks emerging from the vertex. The 
vertex recognition can also be used to check the association of tracks to a vertex, in other words, to determine if a track actually originates from a certain vertex. In order to determine the direction
of the track connecting a primary and a secondary vertex, we have to find the position of the secondary vertex. Some particles can pass through the detector without leaving tracks. However,
when these undetected particles decay, the particles produced can be observed because they leave tracks on the detector. The point in which the product particles are detected is called a secondary 
vertex, and it is said that it is a displaced vertex. Figure \ref{Displaced_vertex} shows a sketch of a displaced vertex, where the path of the undetected particle is represented by a dotted line.

 % Imagen tomada Displaced Vertex: https://indico.cern.ch/event/149404/contributions/1388848/attachments/149638/211969/azuma_berkeley_111020_v05.pdf
 % Imagen Impact parameter: http://www.quantumdiaries.org/2011/06/10/to-b-or-not-to-bbar-b-tagging-via-track-counting/
  
 \begin{figure}[h] 
 \centering
 \caption{Scheme of a displaced vertex. Image taken from \cite{{Displaced_vertex_image}}}
 \includegraphics[width=0.4\textwidth]{./Capitulos/VariableDefinitions/Displaced_vertex} 
 \label{Displaced_vertex}
 \end{figure}
 
The impact parameter is defined as the closest distance between the vertex and the points of the track. A visualization of this is showed in Figure \ref{Impact_parameter}. In this image, the track is represented by the blue dotted line, and the impact parameter by the red line. It can be seen that the impact parameter line forms a right angle with the track. Using this characteristic it is possible to identify in a unique way the closest point of approach of the track to the vertex.


 \begin{figure}[h] 
 \centering
 \caption{Scheme of the impact parameter variable. Image taken from \cite{Impact_parameter_image}}
 \includegraphics[width=0.7\textwidth]{./Capitulos/VariableDefinitions/impactParameter}  
 \label{Impact_parameter}
 \end{figure} 

 
 
 
 
 
 
 
 
 
 
 
 
 
 
 
 
 
 
 
\chapter{CMS Detector}

\section{Tracks Detector}

\section{Hadronic Calorimeter}

\section{Electromagnetic Calorimeter}

\section{Muon Detector}

\section{Triggers}
%\thispagestyle{empty}
 \chapter{Model and backgrounds}

 
\section{Signal of Interest}

The model that is going to be study is based on a reciently proposed new mechanism of production of heavy neutrinos through the Higgs Boson decay \cite{Seesaw Mechanism with displaced vertices}. One favourable characteristic of this model is that in a natural escenario the mass of the heavy neutrinos can lie at the electroweak scale. The theoretical study by \cite{Seesaw Mechanism with displaced vertices} propose the experimental search of the heavy neutrinos using a technique known as displaced vertices.

According to this model when the mass of the heavy neutrinos is inferior than the mass of the Higgs, the latter can present novel decay channels. The Higgs boson can decay into a light and a heavy neutrino, followed by a subsequent decay of the heavy neutrino via a charged or neutral current interaction. Then, the decays of the heavy neutrino can be represented by: $N \rightarrow l^+ l^- \nu$ or $N \rightarrow l q q'$. Thus, there are two possible final states of the event of interest: two leptons, two jets (from the VBF process) and $\vec{E_T^{miss}}$ (due to the neutrino) or four jets (two of the VBF process and two from the quarks of the heavy neutrino decays), $\vec{E_T^{miss}}$ and one lepton. The first type of final state is going to be called hadron signal, while the second will be named leptonic signal.

If the heavy neutrino would have a mass of the order of a few GeV, the Higgs and heavy neutrino would travel a certain distance before decaying. Since both particles are not detected, the decay products are expected to have associated tracks with displaced vertices. For this reason, the presence of displaced vertices in the detector is an important signal to prove this model because it could indicates the presence of the heavy neutrino in the detector. Nevertheless, due to in this model the resulting leptons have a low momentum. Thud, due to experimental restrictions of the available triggers in CMS and ATLAS, the theoretical analysis proposed in reference \cite{Seesaw Mechanism with displace vertices} is not achievable.  

The High Energy Physics Group at Universidad de los Andes has proposed a technique that allows to study at the LHC the production of heavy neutrinos through the decay of the Higgs boson. While in the model proposed in \cite{Seesaw Mechanism with displace vertices} consider the production of the Higgs boson through the fusion of two gluons, we consider the Higgs production through two vector bosons fusion. These vector bosons ($W^{\pm},Z^0,\gamma$) come from an interaction process between two quarks. These both quarks belong to protons from opposite beams that will collide in a particle accelerator. The former described process is know as Vector Boson Fusion (VBF) \cite{VBF processes}. 

Finally, as a result of the two vector boson fusion, a Higgs boson is produced and the initial quarks that interacted manifest themselves as jets with high transverse momentum in the opposite hemispheres of the detector. For this reason, in Experimental Particle Collider Physics the events in which two jets of high transverse momentum are detected located at opposited hemispheres of the detector and with a high separation of pseudorapidity are labeled as candidates of processes of VBF. The feynman diagram ilustrating the process already described in which just one lepton is produced is showed in figure....

\begin{figure}[h] \label{Signal_feynman}
\centering
\caption{Feynman diagram for the hadronic signal}
\includegraphics[width=0.6\textwidth]{./Capitulos/Model/signal}  
\end{figure}

%The observation of the Higgs decay into heavy neutrinos would be a firm prove of the Type I Seesaw mechanism \cite{Type I Seesaw Mechanism}. The former would prove the existence of physics beyond the SM associated to the mass of the neutrinos.
 
 
 \section{Backgrounds}
 
The main problem of detecting an event of interest is that the magnitude of its signal is significantly smaller with respect to some processes from the SM. For this reason, the processes from the SM that have the same or similar final states as the signal are called backgrounds. Thus, it is fundamental to develop procedures in order to reduce the experimental background under the magnitude of the searched signal. These procedures usually use different variables that exploit the topology of the event and its kinematic characteristics. For example, one of the most famous variables is missing transverse energy ($\vec{E_T^{miss}}$), it is defined as the minus sum of the transverse momentum of the particles that are detected (visible particles). When a set of variables that potentially separate the signal from the background is determined, it is necessary to find the optimal values of them that allow to reduce as much as possible the background. 
 
 \subsection{W + Jets Background}
  
 \subsection{Drell Yan + Jets Background}
 
 \subsection{$t \bar{t}$ Background}
 
 
%\thispagestyle{empty}
\chapter{Methodology} 
\label{Methodology_chapter}

The objective of this project was to make a phenomenological study that allow the identification of a signal with the presence of a heavy neutrino in the experiments of the LHC. For this reason, 
the proposed methodology consisted in the use of different computational programs to simulate the signal and its background as it should be produced and measured at the CMS detector. Next, this data must go through a statistical analysis. The programs that were used to simulate the signal were MadGraph \cite{MadGraph 1, MadGraph 2} and Pythia \cite{Pythia}. Then, the program Delphes is used to simulate the behavior of the multi-purpose CMS detector \cite{Delphes}. Lastly, the statistical study of the data was developed with the software ROOT \cite{Root}, which determined the potential variables that could differentiate the signal and background. In the next paragraphs each program is going to be described, including the fundamental physical basis on which the program is constructed and its specific task in the development of the project.

\section{MadGraph}

The first program that was used is MadGraph, which is a generator of events that simulates the collisions of particle beams, which in our case are protons. MadGraph is written in Python programming language. The physical processes that MadGraph can simulate include processes from the SM and from physics beyond the SM that are based on certain theoretical models such as Supersymmetry. This program incorporates diverse physical parameters in order to include all the necessary elements to make phenomenological studies: it calculates the cross section of a certain event, it generates events with strong interactions (including possible decay of particles) and it offers relevant tools to manipulate the events and to make their posterior analysis. 

Madgraph uses perturbation theory to perform production calculations and to generate physical processes. The parameter entries are controlled in configuration files that are called input cards. 
These cards are use to modify essential variables in the production of the events, for example: the type of particles that will collide, the energy of the collision, number of events that are going to be simulated, mass of the generated particles, final states, among others. At the level of event generation it is possible to make basic cuts of minimal and maximal values of some kinematic variables. Moreover, the lastest version of MadGraph (MadGraph 5) has an useful characteristic: it can give an output file with matrix elements that can be used directly in the program Pythia. 

In order to produce an event of physics beyond the SM one has to describe the physical model in the form of a Lagrangian , a list of fields and parameters. Then use the former elements as parameters input of the MATHEMATICA-based package FEYNRULES. Finally, FEYNRULES returns the Feynman rules corresponding to the Lagrangian of the model, which are used as input of MadGraph.


\section{Pythia}

The second computational program that was used is called Pythia. This program receives as parameter input the file generated by the software MadGraph. Pythia incorporates a set of physical models to develop the evolution of a few-body system into a complex multi-particle final state. Thus, the task of the Pythia in the project was to simulate the processes of hadronization of quarks and gluons.

%EXPANDIR MAAAAAAAAAAAAAAAAAAAAAAAAAAAAS

\section{Delphes}

The next program that was used receives as input the events produced by Pythia and it is called Delphes. This software makes a realistic simulation of the multipurpose CMS detector performance as it would happen if there was occurring such an event at CMS. The simulation includes a system of track reconstruction immersed in a magnetic field, an electromagnetic calorimeter, a hadronic calorimeter and a muon detection system.

Delphes takes into account the systematic errors that can be generated by the detector, which can be caused by multiple factors such as the resolution of the detectors. This program contemplates different characteristics of the event in the experiment: detector geometry, the track of the charged particles in the magnetic field, reconstruction of the events, and efficiencies of the reconstruction and particle identification. Due to that the proposed analysis includes the systematic errors that can be generated by the detector, it can be implemented in the experimental studies at the LHC. 

\section{ROOT}

The analysis of the simulated data was developed using the software ROOT. This software was created by the CERN laboratory. ROOT is written in the programming language C++ and it was 
designed to analyze data in particle physics. This program provides all the necessary tools to efficiently process large data, make statistical analyses, and visualize and store data. The program 
includes a numerous quantity of mathematical and statistic functions, numeric algorithms and methods for analysis of data regression. One key tool ROOT has are the histograms that can even use multidimensional data and estimate their density. The histograms can be manipulated, offer statistical information and can make data regression. 

The program ROOT receives as input parameter complementary information that allows it to do the best analysis of the signal: characteristics of the detector or configuration settings that were made in the simulations. ROOT includes other components like a command interpreter that makes quicker the analysis process and a graphic interface which contains a flexible set of tools. The former means that the set of tools can be modified using GUI Builder (the graphic interface constructor). This software can be used to analyse real or simulated data that have the same structure and consist of many events. 
%\thispagestyle{empty}
\chapter{Event Selection Criteria}
\label{Event_selection_criteria_chapter}

The analysis of the simulated data started by imposing a minimum value of 20.0 GeV on $p_T$ for b-jets and taus, and of 30.0 GeV on jets that are not associated to taus or b quarks. The minimum values 
required on $p_T$ are imposed in order to preselect the interesting particles. After this was done, it was required a minimal distance of separation between taus and muons, taus and electrons, 
and taus and jets. The former was done in order to ensure that taus do not overlap with other objects, so the performed analysis does not commit errors such as tagging incorrectly a particle.
To do this, one has to impose a minimal value on the variable called minimal separation distance $\Delta R$, which is given by the Equation \ref{minimal_separation}. The minimal value imposed on this 
variable was 0.3 for taus with muons, electrons and jets. 

\begin{equation}
\Delta R \equiv \sqrt{\Delta \eta ^2 + \Delta \phi ^2}
\label{minimal_separation}
\end{equation}

Since it was considered that the event of interest is generated by a process of VBF, in the analysis was necessary to find the two possible jets that are related to the two VBF jets. This was done by 
searching two jets that satisfied the conditions of having a minimal value on $p_T$ of 50.0 GeV and $\eta$ greater that 5.0 each. The condition of minimal values on $p_T$ and $\eta$ is required 
because it is expected that the VBF jets have a large momentum and that they are located at the end-caps of the detector. Then, an algorithm was performed to select the pair of jets that had the 
largest sum of the invariant mass of all possible combinations of jets pairs in each event. Finally, a minimal value of 100 GeV is required for the sum of the jets. That is how the expected two jets from the VBF 
process are determined. From both jets, the one with the hightest momentum is referred as the leading jet, and the other is known as the subleading jet. The required values to identify the VBF 
jets are part of the preselection process.

After the former preselection process is made, we have to find the minimal or maximum values of variables that allow to reduce the background to its minimum. These minimal and maximum
required values are known as cuts. The first cut imposed in the analysis was on the number of jets. It was required a maximum number of 5 jets because the signal is expected to have 4 jets. An
additional jet is included in order to find the best two VBF jets. The second cut imposed was the presence of just one tau, due to the fact that this is a characteristic expected for the hadronic 
signal. 

Next, the third cut imposed was requiring the number of b-jets to be zero. The former is done to reduce drastically the $t\overline{t}$ background because it has in the final state b-jets. The 
following cut made was on the variable $\vec{E_T^{miss}}$: it was required a value greater that 20 MeV. The former is motivated to reduce the W+Jets background. Then, a cut on the absolute value of $\eta$ of the tau of maximum 2.1 was performed. This value was 
chosen because the tracking detector covers the range of $|\eta|<2.5$. Since the reconstruction cones of the taus have a radius of around 0.4, one has to impose a maximum value on $|\eta|$ of 
maximum 2.5 - 0.4 = 2.1 so the reconstruction algorithms are correctly applied in the area covered by the track detector.
  
After a cut on the $\vec{E}_T^{miss}$ variable was imposed, the cuts motivated by the VBF proccess were performed. The first cut guarantees that there are minimum two jets that satisfy the 
condition on $p_T$ to be candidates of VBF jets. Next, it was required that the product of eta between both jets is negative. This implies that both jets are located at the opposite hemispheres,
as it is expected for both of the VBF jets. Next, it was imposed a minimal value of the diference in $\eta$ for both jets, referred as $\Delta \eta$. The minimal value on $\Delta \eta$ was required 
to be 3.8, since the VBF jets should have a large difference in the pseudorapidity value. 


Finally, a cut in the sum of the invariant mass of the VBF jets, known as dijet mass, was made. It was imposed a 
value of minimum 500.0 MeV on the dijet mass, due to the fact that the VBF jets have a large momentum, and this is proportional to the mass.  

The table \ref{preselection_table} shows the preselection values imposed that were mentioned on the initial paragraphs of this chapter. Additionally, the table \ref{Cuts_variables} shows all 
the cuts that were performed on the data in order to reduce as much as possible the backgrounds. 

%tabla preselection values
\begin{table}[h]
\centering
\caption{Preselection criteria}
\label{preselection_table}
\begin{tabular}{|c|c|}
\hline
Variables                                                                & Value                 \\ \hline
$p_T(b-jets)$ \& $p_T(\tau)$                                            & \textgreater 20.0 GeV \\ \hline
$p_T(jets)$                                                             & \textgreater 30.0 GeV  \\ \hline
$\Delta R (\tau, e)$, $\Delta R (\tau, \mu)$ \& $\Delta R (\tau, jets)$ & \textgreater 0.3      \\ \hline
$P_T(VBF\_jet)$                                                          & \textgreater 50.0 GeV \\ \hline
$\eta(VBF\_jet)$                                                         & \textgreater5.0       \\ \hline
di-Jet Mass                                                               & \textgreater100.0 GeV \\ \hline
\end{tabular}
\end{table}

%tabla cortes

\begin{table}[h]
\centering
\caption{Cuts on different variables}
\label{Cuts_variables}
\begin{tabular}{|c|c|}
\hline
Variables                & Values                \\ \hline
n(jets)                  & \textless 5.0         \\ \hline
n($\tau$)                & = 1                   \\ \hline
n(b-jets)                & = 0                   \\ \hline
$|\eta(\tau)|$           & \textless 2.1         \\ \hline
$\vec{E_T^{miss}}$       & \textgreater 20.0 GeV \\ \hline
n($p_T(jet) > 50.0$)     & $\geq 2$              \\ \hline
$\eta(jet_l)\eta(jet_s)$ & \textless0            \\ \hline
$\Delta \eta$            & \textgreater3.8       \\ \hline
di-Jet Mass                & \textgreater500 GeV    \\ \hline
\end{tabular}
\end{table}



 \chapter{Analysis}

 \chapter{Event Selection Criteria}
%\thispagestyle{empty}
\chapter{Conclusions} 
\label{Conclusion_chapter}

%\thispagestyle{empty}

%----------------------------------------------------------------------------------------
%	APÉNDICES
%----------------------------------------------------------------------------------------

%\addtocontents{toc}{\vspace{2em}} % Agrega espacios en la toc

%\appendix % Los siguientes capítulos son apéndices
\begin{appendices}
%  Incluye los apéndices en el folder de apéndices

%\chapter{Neutrinos and Seesaw Mechanism} \label{apendice neutrinos}

 
\thispagestyle{empty}
\chapter{Neutrinos and Seesaw Mechanism} 
\label{apendice_neutrinos}

%DEFINIR OPERADORES P_R Y P_L

First of all we are going to start by defining some fundamental concepts: helicity, quirality and projection operators. The helicity of a particle is defined as the projection of its spin onto the direction of its motion. It is said that a particle is right-handed when its spin is in the same direction as its motion and it is said a particle is left-handed when its spin is opposite in the opposite direction of its motion. In the case of massless particles the concept of quirality and helicity is equivalent. The quirality for a Dirac fermion is defined through the operator $\gamma^5$ with eigenvalues $\pm 1$. Thus a Dirac field can be projected into its left or right component by acting the operators $P_R$ and $P_L$ upon it. The right- and left-handed projection operators are defined as:
\begin{equation}
P_R = \frac{1 + \gamma^5}{2} \ \ \text{and } \ \ P_L = \frac{1 - \gamma^5}{2}
\end{equation}


\subsection{Dirac Mass}
In this Appendix we are going to perform with detail the calculations for neutrino physics which were mentioned in the State of the Art Chapter. 
We start here by studying the Dirac Mass, which is a term of the form:

\begin{equation}
 m \overline{\psi} \psi = m \overline{(\psi_L + \psi_R)} (\psi_L + \psi_R) = m(\overline{\psi_L} \psi_L + \overline{\psi_L}\psi_R + \overline{\psi_R}\psi_L + \overline{\psi_R} \psi_R)
\end{equation}

Lets study the term $\overline{\psi_L}\psi_L$ and using $P_R  P_L = 0$:

\begin{equation}
 \overline{\psi_L}\psi_L = \overline{\psi} {P}^{\dagger}_L P_L \psi = \overline{\psi}P_R  P_L \psi = 0
\end{equation}

Using an analogous reasoning we can find $\overline{\psi_R}\psi_R = 0$, too. Finally, we obtain the expresion:

\begin{equation}
  m \overline{\psi} \psi = m (\overline{\psi_L} \psi_R + \overline{\psi_R}\psi_L)
\end{equation}

\subsection{Majorana Mass}

The expression we had for the Dirac Lagrangian was: 
%REVISARRRRRRRRRRRRRRR
\begin{align}
  %\phantom{i = j = k}
  &\begin{aligned}
    \mathllap{L} &= \overline{\psi} \left( i \gamma ^\mu \partial_{\mu} - m \right) \psi \\
    \mathllap{}  &= (\overline{\psi_L} + \overline{\psi_R})( i \gamma ^\mu \partial_{\mu} - m)(\psi_L + \psi_R) \\
    \mathllap{}\label{Expresion_Dirac}  &= i \overline{\psi_L} \gamma ^\mu \partial_{\mu} \psi_L + i \overline{\psi_L} \gamma ^\mu   \partial_{\mu} \psi_R - m \overline{\psi_L} \psi_L - m \overline{\psi_L} \psi_R \\
  &\qquad + i \overline{\psi_R} \gamma ^\mu \partial_{\mu} \psi_L + i \overline{\psi_R} \gamma^{\mu} \partial_{\mu} \psi_R - m \overline{\psi_R} \psi_L - m \overline{\psi_R} \psi_R \\
  \end{aligned}
\end{align}

We already proved that $\overline{\psi_L}\psi_L = \overline{\psi_R}\psi_R = 0$. Now we are going to study the second term in the Equation \ref{Expresion_Dirac}, which has a term of the form:

\begin{align}
  &\begin{aligned}
     \mathllap{P_R\gamma^\mu}  &= \frac{1}{2} (1 + \gamma^5)\gamma^\mu = \frac{1}{2} (\gamma^\mu + \gamma^5 \gamma^\mu) \\        
     \mathllap{}            &= \frac{1}{2} (\gamma^\mu - \gamma^\mu \gamma^5) & \text{Since $\{\gamma^5,\gamma^\mu\} = \gamma^5\gamma^\mu + \gamma^\mu \gamma^5 = 0$} \\
     \mathllap{}            &= \frac{1}{2} \gamma^\mu (1 - \gamma^5) = \gamma^\mu P_L 
  &\end{aligned}
\end{align}

Using what we have found in the last expression, we get for the second term:
\begin{align}
  &\begin{aligned}
     \mathllap{i\overline{\psi_L}\gamma^\mu \partial_\mu \psi_R}  &=  i \overline{\psi}P_R \gamma^\mu \partial_\mu P_R \psi\\        
     \mathllap{}            &=  i \overline{\psi}\gamma^\mu P_L \partial_\mu P_R\psi \\
     \mathllap{}            &=  i \overline{\psi}\gamma^\mu \partial_\mu P_L P_R \psi && \text{Since $P_L$ is a constant operator}\\
     \mathllap{}            &=  0
  &\end{aligned}
\end{align}

Following a similar calculation we get: $i\overline{\psi_R}\gamma^\mu \partial_\mu \psi_L$ = 0. Our next step is to find the two coupled Dirac equations using the Euler-Lagrange equation. We 
obtained for the Lagrangian:   

\begin{equation}
L = i \overline{\psi_R} \gamma^\mu \partial_\mu \psi_R + i \overline{\psi_L}\gamma^\mu \psi_L -m\overline{\psi_R}\psi_L - m \psi_L\psi_R
\end{equation}

Replacing in the Euler-Lagrange equation, we get for both states:
\begin{align}
  &\begin{aligned}\label{major start}
     \mathllap{\frac{\partial L}{\partial(\partial \overline{\psi_R})}}  &=  \frac{\partial L}{\partial \overline{\psi_R}} \ \ \rightarrow \ \ 0 = i \gamma^\mu \partial_\mu \psi_L -m \psi_R  \\        
     \mathllap{\frac{\partial L}{\partial(\partial \overline{\psi_L})}}  &=  \frac{\partial L}{\partial \overline{\psi_L}} \ \ \rightarrow \ \ 0 = i \gamma^\mu \partial_\mu \psi_R -m \psi_L\\ 
  &\end{aligned} 
\end{align}

Now, we are going to find an expression for $\psi_R$ in terms of $\psi_L$. First, we take the hermitian conjugate of the bottom equation in \ref{major start}:

\begin{align}
  &\begin{aligned}
     \mathllap{i \gamma^\mu \partial_\mu \psi_R} &= m\psi_L\\        
     \mathllap{(i \gamma^\mu \partial_\mu \psi_R)^\dagger} &= m\psi_L^\dagger  & \text{Taking the hermitian conjugate} \\
     \mathllap{-i \partial_\mu \psi_R^\dagger \gamma^{\mu \dagger}} &= m \psi_L^\dagger \\
     \mathllap{-i \partial_\mu \psi_R^\dagger \gamma^{\mu \dagger}\gamma^0} &= m \psi_L^\dagger \gamma^0  &\text{Multiplying on the right by $\gamma^0$}\\
     \mathllap{-i \partial_\mu \psi_R^\dagger \gamma^0 \gamma^\mu} &= m \psi_L^\dagger \gamma^0  &\text{Using $\gamma^{\mu \dagger}\gamma^0 = \gamma^0 \gamma^\mu$ }\\    %\leftarrow \gamma^0(\gamma^0 \gamma^{\mu \dagger}\gamma^0) = \gamma^0(\gamma^\mu$) }\\
     \mathllap{-i \partial_\mu \overline{\psi_R}\gamma^\mu} &= m \overline{\psi_L} &\text{We have $\overline{\psi} = \psi^{\dagger}\gamma^0$}\\
     \mathllap{-i (\partial_\mu \overline{\psi_R}\gamma^\mu)^\intercal} &= m \overline{\psi_L}^\intercal &\text{Taking the transpose}\\
     \mathllap{-i \gamma^{\mu \intercal} \partial_\mu \overline{\psi_R}^\intercal} &= m \overline{\psi_L}^\intercal \\
     \mathllap{-i (-C^{-1}\gamma^\mu C) \partial_\mu \overline{\psi_R}^\intercal} &= m \overline{\psi_L}^\intercal &\text{Using $\gamma^{\mu \intercal}= -C^{-1}\gamma^\mu C$}\\
     \mathllap{i \gamma^\mu \partial_\mu C \overline{\psi_R}^\intercal} &= m C \overline{\psi_L}^\intercal &\text{Multiplying on the left by C}\\
  &\end{aligned}
\end{align}

As we saw previously, for the lastest equation to have a similar structure as the top equation of \ref{major start}, the right-handed component of $\psi$ must be:

\begin{equation}
 \psi_R = C \overline{\psi_L}^\intercal
\end{equation}

Now, we need to prove that $C \overline{\psi_L}^\intercal$ is actually right-handed. To do this we apply the left-handed chiral projection operator $P_L$ on this state and
the result must be zero.

\begin{align}
  &\begin{aligned}
     \mathllap{P_L \left( C \overline{\psi_L}^\intercal \right)} &= C P_L^\intercal \overline{\psi_L}^\intercal &\text{Property of C: $P_LC = C P_L^\intercal$}\\        
     \mathllap{} &= C \left( \overline{\psi_L}P_L \right)^\intercal  
  &\end{aligned}
\end{align}

Now, let us examine the term $\overline{\psi_L}P_L$:
 
%REVISARRRRRRRRRRRRRRR TERMINO

\begin{align}
  &\begin{aligned}
    \mathllap{\overline{\psi_L}P_L} &= (P_L \psi)^\dagger \gamma_0 P_L = \psi^\dagger P_L \gamma_0 P_L\\
    \mathllap{} &= \psi^\dagger \gamma^0 P_R P_L = 0
    &\end{aligned}
\end{align}

Hence $C \overline{\psi_L}^\intercal$ is in fact a right-handed quiral state. 
\\

%Now, let us find the neutrino mass eigenstates by diagonalizing the matrix M:

%\begin{equation}\label{matrix m}
%M = 
%\begin{pmatrix}
%  m_L & m_D \\
%  m_D & m_R  
%\end{pmatrix}
%\end{equation}




%\chapter{Charge Conjugation Operator} \label{appendix_charge_conjugation}
%\include{Apendices/AppendixC}
\end{appendices}

\addtocontents{toc}{\vspace{2em}} % Agrega espacio en la toc


%----------------------------------------------------------------------------------------
%	BIBLIOGRAFÍA
%----------------------------------------------------------------------------------------
%\backmatter
%\nocite{*}
%\bibliographystyle{plain}
%\bibliography{bibliografía.bib} %Aquí ponen el nombre del archivo .bib


\begin{thebibliography}{10}

\bibitem{Neutrino experiment 1 mass} Gonzalez-Garcia, M., Maltoni, M., \& Schwetz, T. (2016). Global analyses of neutrino oscillation experiments. Nuclear Physics B, 908, 199-217. http://dx.doi.org/10.1016/j.nuclphysb.2016.02.0331

\bibitem{Neutrino experiment 2 mass} Balantekin, A. \& Haxton, W. (2013). Neutrino oscillations. Progress In Particle And Nuclear Physics, 71, 150-161. http://dx.doi.org/10.1016/j.ppnp.2013.03.007   
 
\bibitem{Neutrino dark matter candidate 1} Bhupal, P.S., Mohapatra, R.N., and Zhang, Y. (2016). Heavy right-handed neutrino dark matter in left-right models. Retrieved from https://arxiv.org/abs/1610.05738     

\bibitem{Neutrino dark matter candidate 2} Bhupal, P.S., Mohapatra, R.N., and Zhang, Y. (2016). Naturally Stable Right-Handed Neutrino Dark Matter. Retrieved from https://arxiv.org/abs/1608.06266

\bibitem{Lep experiment} Almeida Jr., F., Coutinho, Y., Martins Simñoes, J., Vale, M., \& Wulck, S. (2001). Dirac and Majorana heavy neutrinos at LEP II. The European Physical Journal C, 22(2), 277-281. http://dx.doi.org/10.1007/s100520100798  

\bibitem{CMS experiment}  Gluza, J., Jelinsky, T. (2015). Heavy neutrinos and the pp-> lljj CMS data. Retrieved from http://www.sciencedirect.com/science/article/pii/S0370269315005080 

\bibitem{ATLAS experiment} Aad, G., Abbott, B., Abdallah, J., Abdel Khalek, S., Abdinov, O., \& Aben, R. et al. (2015). Search for heavy Majorana neutrinos with the ATLAS detector in pp collisions at s = 8 $ \sqrt{s}=8 $ TeV. Journal Of High Energy Physics, 2015(7). http://dx.doi.org/10.1007/jhep07(2015)162  

\bibitem{Seesaw Mechanism with displaced vertices} Gago, A., Hernández, P., Jones-Peréz, J., Losada, M., Moreno, A. (2015). Probing the Type I Seesaw Mechanism with Displaced Vertices at the LHC. Retrieved from  https://arxiv.org/abs/1505.05880v2

\bibitem{Type I Seesaw Mechanism} Molinaro, E. (2013). Type I Seesaw Mechanism, Lepton Flavour    	Violation and                                                                                                       Higgs Decays. Retrieved from https://arxiv.org/pdf/1303.5856v1.pdf

\bibitem{Particle_Detectors_Claus} Grupen, C., Shwartz, B., \& Spieler, H. (2011). Particle detectors (1st ed.). Cambridge: Cambridge University Press.

\bibitem{Image_jet_definitions} Kirschenmann, H. (2017). Sketch of pp-collision and resulting collimated spray of particles, a jet. Retrieved from https://phys.org/news/2012-07-jets-cms-energy-scale.html

\bibitem{Tesis_luis_alfredo} HUERTAS, L. (2016). Estudio fenomenológico de búsquedas de nueva fsica en el LHC, mediante la producción de pares de staus en conjunto con un jet de ISR. (Master of Science). Universidad de los Andes.

\bibitem{Data_analysis_techniques} Fruhwirth, R., \& Regler, M. (2000). Data Analysis Techniques for High-energy Physics (Cambridge Monographs on Particle Physics, Nuclear Physics, and Cosmology) (1st ed.). Cambridge University Press.

\bibitem{CMS_ATLAS_coordinates} Schott, M. (2017). Illustration of the ATLAS and CMS coordinate system. Retrieved from https://inspirehep.net/record/1294662/plots

\bibitem{Displaced_vertex_image}  Azuma, Y. SUSY searches with Displaced Vertices (Disappearing Tracks) in ATLAS. Lecture, Berkeley.

\bibitem{Impact_parameter_image} DORNEY, B. (2017). Visualization of the Impact Parameter (IP, red line) of a track (Image courtesy of Jean-Roch Vlimant, of the CMS Collaboration). Retrieved from http://www.quantumdiaries.org/2011/06/10/to-b-or-not-to-bbar-b-tagging-via-track-counting/

\bibitem{CMS_detector_slice} CMS. (2011). Detector overview. Retrieved from http://cms.web.cern.ch/news/detector-overview

\bibitem{Perspectives_LHC} Kane, G., \& Pierce, A. (2008). Perspectives on LHC physics (1st ed.). Singapore [u.a.]: World Scientific.

\bibitem{Muon_drift_tubes_cms} Muon Drift Tubes | CMS Experiment. (2017). Cms.web.cern.ch. Retrieved 17 May 2017, from http://cms.web.cern.ch/news/muon-drift-tubes

\bibitem{CSC_CMS} Cathode Strip Chambers | CMS Experiment. (2017). Cms.web.cern.ch. Retrieved 17 May 2017, from http://cms.web.cern.ch/news/cathode-strip-chambers

\bibitem{RPC_CMS} Resistive Plate Chambers | CMS Experiment. (2017). Cms.web.cern.ch. Retrieved 17 May 2017, from http://cms.web.cern.ch/news/resistive-plate-chambers

\bibitem{LHC_collitions_web} LHC collisions. (2017). Lhc-machine-outreach.web.cern.ch. Retrieved 17 May 2017, from https://lhc-machine-outreach.web.cern.ch/lhc-machine-outreach/collisions.htm

\bibitem{VBF processes} Dutta, B., Gurrola, A., Johns, W., Kamon, T., Sheldon, P., \& Sinha, K. (2013). Vector boson fusion processes as a probe of supersymmetric electroweak sectors at the LHC. Physical Review D, 87(3). http://dx.doi.org/10.1103/physrevd.87.035029  

\bibitem{MadGraph 1} Alwall, J., Herquet, M., Maltoni, F., Mattelaer, O., \& Stelzer, T. (2011).   MadGraph 5: going beyond. Journal Of High Energy Physics, 2011(6). http://dx.doi.org/10.1007/jhep06(2011)128  

\bibitem{MadGraph 2} Alwall, J., Frederix, R., Frixione, S., Hirschi, V., Maltoni, F., \& Mattelaer, O. et al. (2014). The automated computation of tree-level and next-to-leading order differential cross sections, and their matching to parton shower simulations. Journal Of High Energy Physics, 2014(7). http://dx.doi.org/10.1007/jhep07(2014)079  

\bibitem{Pythia} Sjöstrand, T., Ask, S., Christiansen, J., Corke, R., Desai, N., \& Ilten, P. et al. (2015). An introduction to PYTHIA 8.2. Computer Physics Communications, 191, 159-177. http://dx.doi.org/10.1016/j.cpc.2015.01.024  

\bibitem{Delphes} de Favereau, J., Delaere, C., Demin, P., Giammanco, A., Lemaître, V., Mertens, A., \& Selvaggi, M. (2014). DELPHES 3: a modular framework for fast simulation of a generic collider experiment. Journal Of High Energy Physics, 2014(2). http://dx.doi.org/10.1007/jhep02(2014)057  

\bibitem{Root} Antcheva, I., Ballintijn, M., Bellenot, B., Biskup, M., Brun, R., \& Buncic, N. et al.   (2009). ROOT — A C++ framework for petabyte data storage, statistical analysis and visualization. Computer Physics Communications, 180(12), 2499-2512. http://dx.doi.org/10.1016/j.cpc.2009.08.005   


\end{thebibliography}

\end{document}