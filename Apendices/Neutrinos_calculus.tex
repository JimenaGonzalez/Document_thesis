\chapter{Neutrinos and Seesaw Mechanism} \label{apendice_neutrinos}


\subsection{Dirac Mass}
In this Appendix we are going to perform with detail the calculations for neutrino physics which where described in the State of the Art chapter. 
We start here by studying the Dirac Mass wich was a term of the form:

\begin{equation}
 m \overline{\psi} \psi = m \overline{(\psi_L + \psi_R)} (\psi_L + \psi_R) = m(\overline{\psi_L} \psi_L + \overline{\psi_L}\psi_R + \overline{\psi_R}\psi_L + \overline{\psi_R} \psi_R)
\end{equation}

Lets study the term $\overline{\psi_L}\psi_L$ and using $P_R  P_L = 0$:

\begin{equation}
 \overline{\psi_L}\psi_L = \overline{\psi} {P}^{\dagger}_L P_L \psi = \overline{\psi}P_R  P_L \psi = 0
\end{equation}

Using an analogous reasoning we can find $\overline{\psi_R}\psi_R = 0$, too. Finally, we obtain the expresion:

\begin{equation}
  m \overline{\psi} \psi = m (\overline{\psi_L} \psi_R + \overline{\psi_R}\psi_L)
\end{equation}

\subsection{Majorana Mass}

The expresion we had for the Dirac Lagrangian was: 

\begin{align}
  %\phantom{i = j = k}
  &\begin{aligned}
    \mathllap{L} &= \overline{\psi} \left( i \gamma ^\mu \partial_{\mu} - m \right) \psi \\
    \mathllap{}  &= (\overline{\psi_L} + \overline{\psi_R})( i \gamma ^\mu \partial_{\mu} - m)(\psi_L + \psi_R) \\
    \mathllap{}  &= i \overline{\psi_L} \gamma ^\mu \partial_{\mu} \psi_L + i \overline{\psi_L} \gamma ^\mu   \partial_{\mu} \psi_R - m \overline{\psi_L} \psi_L - m \overline{\psi_L} \psi_R \\
  &\qquad + i \overline{\psi_R} \gamma ^\mu \partial_{\mu} \psi_L + i \overline{\psi_R} \gamma^{\mu} \partial_{\mu} \psi_R - m \overline{\psi_R} \psi_L - m \overline{\psi_R} \psi_R \\
  \end{aligned}\\
\end{align}

We already proved that $\overline{\psi_L}\psi_L = \overline{\psi_R}\psi_R = 0$. Now we study the second term in the latest equation, which has a term of the form:
\begin{align}
  &\begin{aligned}
     \mathllap{P_R\gamma^\mu}  &= \frac{1}{2} (1 + \gamma^5)\gamma^\mu = \frac{1}{2} (\gamma^\mu + \gamma^5 \gamma^\mu) \\        
     \mathllap{}            &= \frac{1}{2} (\gamma^\mu - \gamma^\mu \gamma^5) & \text{Since $\{\gamma^5,\gamma^\mu\} = \gamma^5\gamma^\mu + \gamma^\mu \gamma^5 = 0$} \\
     \mathllap{}            &= \frac{1}{2} \gamma^\mu (1 - \gamma^5) = \gamma^\mu P_L 
  &\end{aligned}
\end{align}

Using what we have found in the last expression, we get for the second term:
\begin{align*}
  &\begin{aligned}
     \mathllap{i\overline{\psi_L}\gamma^\mu \partial_\mu \psi_R}  &=  i \overline{\psi}P_R \gamma^\mu \partial_\mu P_R \psi\\        
     \mathllap{}            &=  i \overline{\psi}\gamma^\mu P_L \partial_\mu P_R\psi \\
     \mathllap{}            &=  i \overline{\psi}\gamma^\mu \partial_\mu P_L P_R \psi && \text{Since $P_L$ is a constant operator}\\
     \mathllap{}            &=  0
  &\end{aligned}
\end{align*}

Following a similar calculus we get: $i\overline{\psi_R}\gamma^\mu \partial_\mu \psi_L$ = 0. Our next step is to find the two coupled Dirac equations using the Euler-Lagrange equation. We obtained for the Lagrangian:

\begin{equation}
L = i \overline{\psi_R} \gamma^\mu \partial_\mu \psi_R + i \overline{\psi_L}\gamma^\mu \psi_L -m\overline{\psi_R}\psi_L - m \psi_L\psi_R
\end{equation}

Replacing in the Euler-Lagrange equation we get for both states:
\begin{align*}
  &\begin{aligned}
     \mathllap{\frac{\partial L}{\partial(\partial \overline{\psi_R})}}  &=  \frac{\partial L}{\partial \overline{\psi_R}} \ \ \rightarrow \ \ 0 = i \gamma^\mu \partial_\mu \psi_L -m \psi_R  \\        
     \mathllap{\frac{\partial L}{\partial(\partial \overline{\psi_L})}}  &=  \frac{\partial L}{\partial \overline{\psi_L}} \ \ \rightarrow \ \ 0 = i \gamma^\mu \partial_\mu \psi_R -m \psi_L\\ 
  &\end{aligned} \label{major start}
\end{align*}

Now, we are going to find an expression for $\psi_R$ in terms of $\psi_L$. We take the hermitian conjugate of the equation $\ref{major start}$:

\begin{align*}
  &\begin{aligned}
     \mathllap{(i \gamma^\mu \partial_\mu \psi_R)^\dagger}  &= m\psi_L^\dagger 
  &\end{aligned}
\end{align*}





