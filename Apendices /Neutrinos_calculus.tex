\chapter{Neutrinos and Seesaw Mechanism} \label{apendice_neutrinos}

%DEFINIR OPERADORES P_R Y P_L

\subsection{Dirac Mass}
In this Appendix we are going to perform with detail the calculations for neutrino physics which were mentioned in the State of the Art Chapter. 
We start here by studying the Dirac Mass, which is a term of the form:

\begin{equation}
 m \overline{\psi} \psi = m \overline{(\psi_L + \psi_R)} (\psi_L + \psi_R) = m(\overline{\psi_L} \psi_L + \overline{\psi_L}\psi_R + \overline{\psi_R}\psi_L + \overline{\psi_R} \psi_R)
\end{equation}

Lets study the term $\overline{\psi_L}\psi_L$ and using $P_R  P_L = 0$:

\begin{equation}
 \overline{\psi_L}\psi_L = \overline{\psi} {P}^{\dagger}_L P_L \psi = \overline{\psi}P_R  P_L \psi = 0
\end{equation}

Using an analogous reasoning we can find $\overline{\psi_R}\psi_R = 0$, too. Finally, we obtain the expresion:

\begin{equation}
  m \overline{\psi} \psi = m (\overline{\psi_L} \psi_R + \overline{\psi_R}\psi_L)
\end{equation}

\subsection{Majorana Mass}

The expression we had for the Dirac Lagrangian was: 
%REVISARRRRRRRRRRRRRRR
\begin{align}
  %\phantom{i = j = k}
  &\begin{aligned}
    \mathllap{L} &= \overline{\psi} \left( i \gamma ^\mu \partial_{\mu} - m \right) \psi \\
    \mathllap{}  &= (\overline{\psi_L} + \overline{\psi_R})( i \gamma ^\mu \partial_{\mu} - m)(\psi_L + \psi_R) \\
    \mathllap{}  &= i \overline{\psi_L} \gamma ^\mu \partial_{\mu} \psi_L + i \overline{\psi_L} \gamma ^\mu   \partial_{\mu} \psi_R - m \overline{\psi_L} \psi_L - m \overline{\psi_L} \psi_R \\
  &\qquad + i \overline{\psi_R} \gamma ^\mu \partial_{\mu} \psi_L + i \overline{\psi_R} \gamma^{\mu} \partial_{\mu} \psi_R - m \overline{\psi_R} \psi_L - m \overline{\psi_R} \psi_R \\
  \end{aligned}\\
\end{align}

We already proved that $\overline{\psi_L}\psi_L = \overline{\psi_R}\psi_R = 0$. Now we are going to study the second term in the latest (REFERENCIAR) equation, which has a term of the form:

\begin{align}
  &\begin{aligned}
     \mathllap{P_R\gamma^\mu}  &= \frac{1}{2} (1 + \gamma^5)\gamma^\mu = \frac{1}{2} (\gamma^\mu + \gamma^5 \gamma^\mu) \\        
     \mathllap{}            &= \frac{1}{2} (\gamma^\mu - \gamma^\mu \gamma^5) & \text{Since $\{\gamma^5,\gamma^\mu\} = \gamma^5\gamma^\mu + \gamma^\mu \gamma^5 = 0$} \\
     \mathllap{}            &= \frac{1}{2} \gamma^\mu (1 - \gamma^5) = \gamma^\mu P_L 
  &\end{aligned}
\end{align}

Using what we have found in the last expression, we get for the second term:
\begin{align}
  &\begin{aligned}
     \mathllap{i\overline{\psi_L}\gamma^\mu \partial_\mu \psi_R}  &=  i \overline{\psi}P_R \gamma^\mu \partial_\mu P_R \psi\\        
     \mathllap{}            &=  i \overline{\psi}\gamma^\mu P_L \partial_\mu P_R\psi \\
     \mathllap{}            &=  i \overline{\psi}\gamma^\mu \partial_\mu P_L P_R \psi && \text{Since $P_L$ is a constant operator}\\
     \mathllap{}            &=  0
  &\end{aligned}
\end{align}

Following a similar calculation we get: $i\overline{\psi_R}\gamma^\mu \partial_\mu \psi_L$ = 0. Our next step is to find the two coupled Dirac equations using the Euler-Lagrange equation. We 
obtained for the Lagrangian:   

\begin{equation}
L = i \overline{\psi_R} \gamma^\mu \partial_\mu \psi_R + i \overline{\psi_L}\gamma^\mu \psi_L -m\overline{\psi_R}\psi_L - m \psi_L\psi_R
\end{equation}

Replacing in the Euler-Lagrange equation, we get for both states:
\begin{align}
  &\begin{aligned}\label{major start}
     \mathllap{\frac{\partial L}{\partial(\partial \overline{\psi_R})}}  &=  \frac{\partial L}{\partial \overline{\psi_R}} \ \ \rightarrow \ \ 0 = i \gamma^\mu \partial_\mu \psi_L -m \psi_R  \\        
     \mathllap{\frac{\partial L}{\partial(\partial \overline{\psi_L})}}  &=  \frac{\partial L}{\partial \overline{\psi_L}} \ \ \rightarrow \ \ 0 = i \gamma^\mu \partial_\mu \psi_R -m \psi_L\\ 
  &\end{aligned} 
\end{align}

Now, we are going to find an expression for $\psi_R$ in terms of $\psi_L$. First, we take the hermitian conjugate of the bottom equation in \ref{major start}:

\begin{align}
  &\begin{aligned}
     \mathllap{i \gamma^\mu \partial_\mu \psi_R} &= m\psi_L\\        
     \mathllap{(i \gamma^\mu \partial_\mu \psi_R)^\dagger} &= m\psi_L^\dagger  & \text{Taking the hermitian conjugate} \\
     \mathllap{-i \partial_\mu \psi_R^\dagger \gamma^{\mu \dagger}} &= m \psi_L^\dagger \\
     \mathllap{-i \partial_\mu \psi_R^\dagger \gamma^{\mu \dagger}\gamma^0} &= m \psi_L^\dagger \gamma^0  &\text{Multiplying on the right by $\gamma^0$}\\
     \mathllap{-i \partial_\mu \psi_R^\dagger \gamma^0 \gamma^\mu} &= m \psi_L^\dagger \gamma^0  &\text{Using $\gamma^{\mu \dagger}\gamma^0 = \gamma^0 \gamma^\mu$ }\\    %\leftarrow \gamma^0(\gamma^0 \gamma^{\mu \dagger}\gamma^0) = \gamma^0(\gamma^\mu$) }\\
     \mathllap{-i \partial_\mu \overline{\psi_R}\gamma^\mu} &= m \overline{\psi_L} &\text{We have $\overline{\psi} = \psi^{\dagger}\gamma^0$}\\
     \mathllap{-i (\partial_\mu \overline{\psi_R}\gamma^\mu)^\intercal} &= m \overline{\psi_L}^\intercal &\text{Taking the transpose}\\
     \mathllap{-i \gamma^{\mu \intercal} \partial_\mu \overline{\psi_R}^\intercal} &= m \overline{\psi_L}^\intercal \\
     \mathllap{-i (-C^{-1}\gamma^\mu C) \partial_\mu \overline{\psi_R}^\intercal} &= m \overline{\psi_L}^\intercal &\text{Using $\gamma^{\mu \intercal}= -C^{-1}\gamma^\mu C$}\\
     \mathllap{i \gamma^\mu \partial_\mu C \overline{\psi_R}^\intercal} &= m C \overline{\psi_L}^\intercal &\text{Multiplying on the left by C}\\
  &\end{aligned}
\end{align}

As we saw previously, for the lastest equation to have a similar structure as the top equation of \ref{major start}, the right-handed component of $\psi$ must be:

\begin{equation}
 \psi_R = C \overline{\psi_L}^\intercal
\end{equation}

Now, we need to prove that $C \overline{\psi_L}^\intercal$ is actually right-handed. To do this we apply the left-handed chiral projection operator $P_L$ on this state and
the result must be zero.

\begin{align}
  &\begin{aligned}
     \mathllap{P_L \left( C \overline{\psi_L}^\intercal \right)} &= C P_L^\intercal \overline{\psi_L}^\intercal &\text{Property of C: $P_LC = C P_L^\intercal$}\\        
     \mathllap{} &= C \left( \overline{\psi_L}P_L \right)^\intercal  
  &\end{aligned}
\end{align}

Now, let us examine the term $\overline{\psi_L}P_L$:
 
%REVISARRRRRRRRRRRRRRR TERMINO

\begin{align}
  &\begin{aligned}
    \mathllap{\overline{\psi_L}P_L} &= (P_L \psi)^\dagger \gamma_0 P_L = \psi^\dagger P_L \gamma_0 P_L\\
    \mathllap{} &= \psi^\dagger \gamma^0 P_R P_L = 0
    &\end{aligned}
\end{align}

Hence $C \overline{\psi_L}^\intercal$ is in fact a right-handed quiral state. 
\\

%Now, let us find the neutrino mass eigenstates by diagonalizing the matrix M:

%\begin{equation}\label{matrix m}
%M = 
%\begin{pmatrix}
%  m_L & m_D \\
%  m_D & m_R  
%\end{pmatrix}
%\end{equation}




